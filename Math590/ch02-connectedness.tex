\chapter{Connectedness}

\section{Paths}
\dfn{Path}{
	Let $X$ be a topological space, and take $x,y\in X$. A \underline{\textbf{path}} from $x$ to $y$ is a continuous map $p:[a,b]\to X$, where $a\leq b\in\mathbb{R}$, such that $p(a) = x$ and $p(b) = y$.
	\begin{itemize}
		\item if $p:[a,b]\to X$ and $q:[b,c]\to X$ are paths such that $p(b)=q(b)$, the \underline{\textbf{concatenation}} is $h:[a,c]\to X$ defined by
		$$\begin{cases}
			t\mapsto p(t)\text{ if } t < b \\
			t\mapsto q(t)\text{ if } t\geq b
		\end{cases}$$
		$h$ is well defined.
	\end{itemize}
}

\begin{note}
	If $p:[a,b]\to X$ is a path from $x$ to $y$, then $q:[0,1]\to X$ defined by $t\mapsto p((1-t)a+tb)$ is also a path from $x$ to $y$.
\end{note}

\thm{The Pasting Lemma}{
	Let $X,Y$ be topological spaces, and let $A,B\subset X$ be closed such that $X=A\cup B$. Let $f:A\to Y$ and $g:B\to Y$ be continuous such that $f(x)=g(x)\forall x\in A\cap B$.
	Take $h:X\to Y$ to be the function defined by
	$$h(x) = \begin{cases}
		f(x) \text{ if }x\in A \\
		g(x) \text{ if }x\in B
	\end{cases}$$
	Then $h$ is well-defined and continuous.
}

\begin{note}
	Let $X$ be a topological space and $C\subset X$ a closed subspace. A subset $A\subset C$ is closed in $C$ if and only if $A$ is closed in $X$.

	\begin{myproof}
		$(\implies)$ Let $A\subset C$ be closed in $C$. Since $A$ is closed in $C, \exists D\subset X$ such that $A= D\cap C$, wher $D$, and since closed-ness is stable under intersection, $D\cap C$ is closed in $X$.
		$(\impliedby)$ Let $A\subset C$ be closed in $X$. Then $A = A\cap C$, so $A$ is closed in $C$.
	\end{myproof}
\end{note}

Then, for the proof of the pasting lemma:

\begin{myproof}
	Let $C\subset Y$ be closed. We want to show that $h^{-1}(C)\subset X$ be closed. Note that $$h^{-1}(C) = (h^{-1}(C)\cap A)\cup (h^{-1}(C)\cap B) = f^{-1}(C)\cup g^{-1}(C)$$
	Where $f^{-1}(C)\subset A$ and $g^{-1}(C)\subset B$ are closed in $X$ by the previous fact. Therefore, $h^{-1}(C)$ is closed in $X$ as closed sets are stable under finite union.
\end{myproof}

\cor[]{}{Concatenations of paths are paths.}

\section{Connectedness}

\dfn[]{}{
	Let $X$ be a topological space and $x,y\in X$. Then
	\begin{itemize}
		\item $x$ and $y$ are \underline{\textbf{path-connected in $X$}} if $\exists$ a path in $X$ from $x$ to $y$.
		\item $X$ is path-connected if any two points are path connected
		\item Note that path-connectedness in $X$ is an equivalence relation. The equivalence class of $x\in X$ is the path-connected component of $x$.
	\end{itemize}
}

\begin{myproof}
	Proof of equivalence relation
	\begin{enumerate}
		\item \textbf{Reflexivity}: take the constant path.
		\item \textbf{Symmetry}: If $p:[0,1]\to X$ connects $x$ to $y$, then $q:[0,1]\to X$ defined by $t\mapsto p(1-t)$ connects $y$ to $x$.
		\item \textbf{Transitivity}: If $p[0,1]\to X$ is a path from $x$ to $y$, and $q[0,1]\to X$ is a path from $y$ to $z$, then the concatenation of $p$ and $q$ is a path from $x$ to $z$.
	\end{enumerate}
\end{myproof}

\ex[]{}{In $\mathbb{R}^k$, balls $B(x,r)$ are path-connected. In fact, $B(x,r)$ is \underline{\textbf{convex}}, $\forall y,z\in B(x,r), \forall t\in [0,1], (1-t)y+tz\in B(x,r)$.}

\thm[]{}{
	Let $X$ be a topological space and $p:[0,1]\to X$ a path from $x$ to $y$. Let $U\subset X$ be an open neighborhood of $x$ such that $y\notin U$. Then $\exists t\in[0,1]$ such that $p(t)\in \partial U$ the boundary of $U$.
}

\begin{myproof}
	TODO: insert proof here
\end{myproof}

\dfn{Separation}{
	Let $X$ be a topological space. A \underline{\textbf{separation}} is a pair $(U,V)$ of disjoint non-empty open sets that cover $X$. i.e. it is a pair $(U,X\setminus U)$ where $U$ is a non-empty proper clopen (closed and open) subset.
}

\dfn{Connected}{
	$X$ is connected if there is no separation.
}

\thm{}{If $X$ is path connected, then $X$ is connected.}

\begin{myproof}
	TODO: insert proof here
\end{myproof}

\newpage

\dfn[]{Interval}{
	Let $I\subset\mathbb{R}$. $I$ is an interval if $I = \{a\}, [a,b], (a,b), [a,b), (-\infty, b), (-\infty, \infty), \emptyset,$ etc.
}
\thm[]{}{
	Let $A\subset\mathbb{R}$. The following are equivalent:
	\begin{enumerate}
		\item $A$ is an interval
		\item $A$ is convex
		\item $A$ is path-connected
		\item $A$ is connected
	\end{enumerate}
}

\thm{}{
	Let $X,Y$ be a topological space and let $f:X\to Y$ be continuous. Then 
	\begin{enumerate}
		\item if $X$ is connected, then $f(X)$ is connected.
		\item if $X$ is path-connected, then $f(X)$ is path-connected.
	\end{enumerate}
}

\begin{myproof}
	TODO: insert proof here
\end{myproof}

\begin{myproof}
	TODO: insert proof here
\end{myproof}

\thm{The Intermediate Value Theorem}{
	Let $X$ be a connected topological space and $f:X\to\mathbb{R}$ be continuous. Then $f(X)\subset \mathbb{R}$ is an interval, i.e. $\forall x,y\in X, \forall t\in\mathbb{R}$ if $f(x)\leq t \leq f(y)$ then $\exists z\in X$ such that $f(z)= t$.
}

\begin{myproof}
	TODO: insert proof here
\end{myproof}

\ex[]{}{$Y = \{(t,\sin(\frac{1}{t}))|t>0\}\cup\{0\}\times [-1,1]$ is connected, but not path connected.}

\begin{myproof}
	TODO: Insert proof here (from lecture on 2/12)
\end{myproof}

\thm{}{
	Let $X$ be a topological space and $A\subset X$ be connected. Then $\barA$ is connected. In particular, if $A$ is dense in $X$ (i.e. $\barA = X$) then $X$ is connected.
}

\begin{myproof}
	TODO: insert proof here
\end{myproof}

\thm{}{
	Let $X$ be a topological space and take $x,y\in X$. Suppose $p:[0,1]\to X$ is a path from $x$ to $y$ and $\exists U$ open such that $x\in U$ but $y\not\in U$. Then $\exists t$ such that $p(t)\in \partial U$.
}

\thm{}{
	Let $X, Y$ be $\begin{cases}\text{connected} \\ \text{path-connected}\end{cases}$. Then $X\times Y$ is $\begin{cases}\text{connected} \\ \text{path-connected}\end{cases}$.
}

\begin{myproof}
	TODO: insert proof here
\end{myproof}

\thm{}{
	Let $X$ be a topological space, and let $(A_\alpha)_{\alpha\in I}$ be a collection of (path-)connected subspaces such that $\bigcap_{\alpha\in I}A_\alpha\neq\emptyset$. Then $\bigcup_{\alpha\in I}A_\alpha\neq\emptyset$ is (path-)connected.
}

\begin{myproof}
	TODO: insert proof here
\end{myproof}

\dfn{Connected Elements}{
	Let $X$ be a topological space. Then $x,y\in X$ are \textbf{connected in $X$} if $\exists A\subset X$ a connected subspace such that $x,y\in A$. Note that this is an equivalence relation. The \textbf{connected component} of $x$ is its equivalence class.
}

\begin{myproof}
	TODO: insert proof here
\end{myproof}

\begin{note}
	If $\sim$ is an equivalence relation on $X$, then the equivalence classes form a partition of $X$. $X$ is the disjoint union of all equivalence classes. If two equivalence classes intersect, then they are equal.
\end{note}

\thm{}{
	\begin{enumerate}
		\item If $x,y$ are path connected in $X$, then $x,y$ are connected in $X$.
		\item The path connected component of $x$ is contained in the connected complement of $x$.
		\item The connected component of $x$ is the union of all connected subspaces containing $x$, or equivalently, the largest connected subspace containing $x$.
	\end{enumerate}
}

\dfn[]{}{
	Let $X$ be a topological space and let $x\in X$. $X$ is \textbf{locally (path) connected at $x$} if $\forall N$ open neighborhood of $x$, $\exists N'\subset N$ which is (path) connected. $X$ is \textbf{locally connected} if it is locally connected at every point.
}

\begin{note}
	A topological space being locally connected at $x$ does not mean that there exists a connected open neighborhood. Similarly   connectedness does not imply local connectedness.
\end{note}

\begin{note}
	If $X$ is locally connected, then any open $U\subset X$ is locally connected.
\end{note}


\thm{}{Let $X$ be a topological space and $A\subset X$ be a connected subspace. Then $\barA$ is connected.}

\cor[]{}{If $X$ is a topological space, then any connected component is closed.

Note that \textbf{path} connected components are not necessarily closed.}

\begin{myproof}
	TODO: insert proof here
\end{myproof}

\thm{Locally Connected}{Let $X$ be a locally (path) connected subspace. Then the (path) connected components are clopen.}

\begin{myproof}
	TODO: insert proof here
\end{myproof}

\thm{}{Let $X$ be a locally path connected space. Then $X$ is locally connected, and $\forall x\in X$, the path connected component of $x$ is equal to the connected component of $x$.}

