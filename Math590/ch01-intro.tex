\chapter{Introduction}

\section{The Standard Topology on Euclidean Space}
Topology, from the greek \textit{topos}, meaning "place" or "locality", and \textit{logos} meaning "study", can be thought of as the study of shape. More specifically, the study of how geometric objects behave under continuous deformations.

There are a variety of different (equivalent) approaches to topology, including but not limited open sets, neighborhoods, metrics, convergence of sequences, and continuity of functions. All of the preceding are discussed in this course, but we will rely heavily on the concept of open sets. Before getting to the subject, we review some important fundamentals.

Firstly, we denote the set of real numbers as $\mathbb{R}$, and the set of $d$-tuples as $\mathbb{R}^{d}$. The latter of these sets is sometimes referred to as "Euclidean $d$-space".

\dfn{The (Standard) Inner Product and (Standard) Norm}{Let $x = (x_1,...,x_d), y=(y_1,...,y_d)\in\mathbb{R}$. The standard inner product is a map $f:\mathbb{R}^d\times\mathbb{R}^d\to\mathbb{R}$ defined by
$$f(x,y) = \langle x,y\rangle = x_1y_1+...x_dy_d$$
We then define the (standard) norm as a function $g:\mathbb{R}^d\to\mathbb{R}$ as 
$$g(x) = ||x|| = \sqrt{\langle x,x\rangle} = \sqrt{x_1^2+...x_d^2}$$}
\thm{}{For all $x,x',y,y'\in\mathbb{R}^d$ the following properties regarding the inner product hold
\begin{enumerate}
	\item Bilinearity:
	$$\langle\lambda x+\lambda'x', \mu y+ \mu'y'\rangle = \lambda\mu\langle x,y\rangle + \lambda'\mu\langle x'y\rangle + \lambda\mu' \langle x,y'\rangle + \lambda'\mu'\langle x',y'\rangle$$
	\item Symmetry:
	$$\langle x, y\rangle = \langle y,x\rangle$$
	\item Positivity:
	$$\langle x,x\rangle \geq 0 \text{ with equality if and only if } x=0$$
	\item Cauchy Schwarz Inequality:
	$$|\langle x,y\rangle|\leq ||x|| \,||y||, \text{ with equality if and only if } x \parallel y$$
	\item Triangle Inequality:
	$$||x+y||\leq ||x|| + ||y||, \text{ with equality if and only } y=0 \text{ or } \exists a\geq 0 \text{ s.t. } x=ay$$
\end{enumerate}}

\begin{myproof}
	TODO: prove the above statements from definition
\end{myproof}



\dfn{The (Standard) Metric on $\mathbb{R}^d$}{The (standard) metric on $\mathbb{R}^d$ is a map $d:\mathbb{R}^d\times\mathbb{R}^d\to\mathbb{R}$ such that $\forall x,y\in\mathbb{R}$
$$d(x,y) = ||x-y||$$}

\thm{}{For all $x,y,z\in\mathbb{R}$, the following properties regarding the (standard) metric hold:
\begin{enumerate}
	\item Positivity:
	$$d(x,y)\geq 0\text{ with equality if and only if }x=y$$
	\item Symmetry:
	$$d(x,y) = d(y,x)$$
	\item Triangle Inequality
	$$d(x,z)\leq d(x,y)+d(y,z)$$
\end{enumerate}}
\begin{myproof}
TODO: prove above statements from definition.
\end{myproof}

\dfn{(Standard) Open Ball and Closed Ball}{Let $x\geq 0$ and $R\geq0$. Then the (standard) open ball is the set
$$B(x, R) = \{y\in\mathbb{R}^d| d(x,y)<R\}$$
and the (standard) close ball is the set
$$\barB(x,R) = \{y\in\mathbb{R}^d| d(x,y)\leq R\}$$}
In $\mathbb{R}$, $B(x,R)$ is just the open interval $(x-R, x+R)$. In $\mathbb{R}^2$, $B(x,R)$ is the interior of the circle centered at $x$ of radius $R$. In $\mathbb{R}^3$, $B(x,R)$ is the the interior of the sphere centered at $x$ of radius $R$. The closed counterparts of these sets include the boundaries of the described sets.

\dfn{Neighborhoods in $\mathbb{R}^d$}{Let $U\subset\mathbb{R}^d$ and $x\in U$. Then $U$ is a \textbf{neighborhood} of $x$ if it contains a nonempty ball centered at $x$, i.e. if
$$\exists\epsilon > 0, B(x,\epsilon)\subset U$$
Additionally, $U$ is a (standard) open ball of $\mathbb{R}^d$ if it is a neighborhood of all its points, i.e.
$$\forall y\in U,\exists \epsilon > 0, B(x,\epsilon)\subset U$$}

\begin{note}
	\begin{enumerate}
		\item $\empty$ and $\mathbb{R}$ are open.
		\item Open balls are open.
	\end{enumerate}
\end{note}

\begin{proof}
	TODO: prove the statements in the note
\end{proof}

\thm{}{
	\begin{enumerate}
		\item Let $(V_\alpha)_{\alpha\in I}$ be a (possibly infinite) family of open subsets of $\mathbb{R}^d$. Then $\bigcup_{\alpha\in I}V_\alpha$ is open.
		\item Let $V_1,...,V_n\in\mathbb{R}^d$ be open. Then $V_1\cap...\cap V_n$ is open.
	\end{enumerate}
}

\begin{proof}
	TODO: prove the above statements
\end{proof}

\ex{}{The set $U = \{(x_1,...,x_d)\in\mathbb{R}^d|x_d > 0\}$ is an open subset of $\mathbb{R}^d$.\\
\begin{myproof}
	Let $x\in U$ and set $\epsilon = x_d$. We then want to show that $B(x,\epsilon)\subset U$. Let $y\in B(x,\epsilon)$. Then $|x_d - y_d|\leq ||x-y|| = \sqrt{|x_1-y_1|^2+...+|x_d-y_d^2|}$
\end{myproof}}

\begin{note}
	The collection of open subsets of $\mathbb{R}^d$ is called the Standard Topology on $\mathbb{R}^d$.
\end{note}

\dfn{Convergence of Sequences in $\mathbb{R}^d$}{Let $(x_n)_{n\geq 0}$ be a sequence in $\mathbb{R}^d$. We say $\displaystyle x_n\to_x$ or $\displaystyle\lim_{n\to\infty}x_n=x$ if 
$$\forall\epsilon>0,\exists n_0\in\mathbb{R}\text{ such that }\forall n>n_0, d(x_n,x)<\epsilon$$
Note that $d(x_n,x)<\epsilon$ is equivalent to $x_n\in B(x,\epsilon)$.}

\ex{}{$\displaystyle\lim_{n\to\infty}\frac{1}{n} = 0$\\ TODO: Proof}

\thm{}{Let $x\in\mathbb{R}$ and $(x_n)_{n\geq 0}$ a sequence in $\mathbb{R}^d$. Then
$$\lim_{n\to\infty}x_n = x \iff \forall V \text{ nbhd of }x,\exists n_0\in\mathbb{R}\text{ such that } \forall n>n_0, x_n\in V$$}

\begin{myproof}
	$(\implies)$: Assume $\lim_{n\to\infty}x_n = x$. Let $V$ be a neighborhood of $x$. Then, by the definition of neighborhood, $\exists\epsilon>0$ such that $B(x,\epsilon)\subset V$. Then, by the definition of convergence, $\exists n_0\in\mathbb{R}$ such that $\forall n > n_0, x_n\in B(x,\epsilon)$. Then, since $B(x,\epsilon)\subset V$, we have $\forall n > n_0, x_n\in V$. Then, since $V$ is arbitrary, $\forall V$ a neighborhood of $x, \exists n_0\in\mathbb{R}, \forall n> n_0, x_n\in V$. Therefore,
	$$\lim_{n\to\infty}x_n = x \implies \forall V \text{ nbhd of }x,\exists n_0\in\mathbb{R}\text{ such that } \forall n>n_0, x_n\in V$$

	$(\impliedby)$ Assume $\forall V \text{ nbhd of }x,\exists n_0\in\mathbb{R}$ such that $ \forall n>n_0, x_n\in V$. Let $\epsilon > 0$. We know that $B(x,\epsilon)$ is a neighborhood of $x$. Then, by our assumption, $\exists n_0\in\mathbb{R}$ such that $\forall n>n_0, x\in B(x,\epsilon)$. Therefore, by the definition of convergence, $\lim_{n\to\infty}x_n = x$. Therefore,
	$$\lim_{n\to\infty}x_n = x \impliedby \forall V \text{ nbhd of }x,\exists n_0\in\mathbb{R}\text{ such that } \forall n>n_0, x_n\in V$$
\end{myproof}

\dfn{Continuity of Functions from $\mathbb{R}^d$ to $\mathbb{R}^k$}{A map $f:\mathbb{R}^d\to\mathbb{R}^k$ is \textbf{continuous at $x\in\mathbb{R}^d$} if 
$$\forall\epsilon > 0,\exists\delta > 0 \text{ such that } \forall y\in B(x,\delta), d(f(x),f(y))<\epsilon$$
Note that $d(f(x),f(y))<\epsilon$ is equivalent to $f(y)\in B(f(y),\epsilon)$, so $f$ is continuous if 
$$\forall\epsilon > 0,\exists\delta > 0 \text{ such that } \forall y\in B(x,\delta), f(y)\in B(f(y),\epsilon)$$
Another, equivalent definition is that $f$ is continuous if 
$$\forall\epsilon > 0,\exists\delta > 0 \text{ such that } f(B(x,\delta))\subset B(f(x),\epsilon)$$
We say that $f$ is \textbf{continuous} if it is continuous at every point.}

\ex{}{
\begin{enumerate}
	\item Let $v\in\mathbb{R}^d$ and let $f:\mathbb{R}^d\to\mathbb{R}^d$ defined by the rule $x\mapsto x+v$ is continuous on $\mathbb{R}^d$.\\
	Let $x\in\mathbb{R}^d$ and let $\epsilon>0$. Set $\delta = \epsilon$. We then want to show that $f(B(x,\delta))\subset B(f(x),\epsilon)$. Let $y\in B(x,\delta)$. Then 
	$$d(f(y),f(x)) = ||f(y)-f(x)|| = ||y+v-(x+v)|| = ||y-x||<\delta = \epsilon$$
	Therefore, $f(y)\in B(f(x),\epsilon)$.
	\item TODO: fill in remaining examples
\end{enumerate}
}

\thm{}{Let $f:\mathbb{R}^d\to\mathbb{R}^k$. Then the following statements hold
\begin{enumerate}
	\item $f$ is continuous at $x\in\mathbb{R}^d \iff \forall V$ neighborhood of $f(x), f^{-1}(V)$ is a neighborhood of $x$.
	\item $f$ is continuous on $\mathbb{R}^d \iff \forall U\subset\mathbb{R}^d$ open, $f^{-1}(U)\subset\mathbb{R}^k$ is open.
	\item $f$ is continuous at $x\in\mathbb{R}^d\iff\forall (x_n)_{n\geq 0}$ in $\mathbb{R}^d$ converging to $x$, $f(x_n)\to f(x)$ as $n\to \infty$. 
\end{enumerate}
}

\begin{myproof}
	TODO: complete proof of above statements
\end{myproof}

\dfn{Open Subset}{Let $X\subset\mathbb{R}^d$. A subset $U\subset X$ is \textbf{open in $X$} if $\forall x\in U,\exists\epsilon>0$ such that $B(x,\epsilon)\cap X\subset U$.}

A Topology of $X$, then, is a collection of subsets of $X$ that are open in $X$. This definition will be expanded and made rigorous later in the course.

\ex{Torus}{TODO: insert torus example here (preferable with illustration)}

\dfn{Homeomorphism}{Let $X,Y\subset\mathbb{R}^d$ be subsets. Then $X,Y$ are \textbf{homeomorphic} if $\exists f:X\to Y$ a bijection such that
$$\forall U\subset X, U\text{ is open in } X\iff f(U)\text{ open in } Y$$
The function $f$ is called a \textbf{homeomorphism}.}

\section{General Topologies, Open Sets, and Neighborhoods}

In the previous section, we defined things in terms of $\mathbb{R}^d$, the standard metric, and the standard inner product.

\dfn{Topology}{Let $X$ be a set. A \textbf{topology} on $X$ is a collection $\mcT\subset\mathcal{P}(X)$ of subsets of $X$ called \textbf{open subsets} such that
\begin{enumerate}
	\item $\emptyset, X\in\mathcal{T}$
	\item Stability under unions. For any $(U_{\alpha})_{\alpha\in I}$ of elements of $\mathcal{T}$, $\bigcup_{\alpha\in I}U_{\alpha}$ is also an element of $\mathcal{T}$.
	\item Stability under finite intersection. For any $U_1,...,U_n\in\mathcal{T}$, we have $U_1\cap...\cap U_n\in\mathcal{T}$.
\end{enumerate}
The pair $(X,d)$ is called a \textbf{topological space}.
}

\ex{}{Let $X=\mathbb{R}^d$.
\begin{enumerate}
	\item The collection of standard open sets is a topology (the standard topology) as discussed in the previous section.
	\item $\{\emptyset, \mathbb{R}^d\}$ is a topology on $\mathbb{R}^d$.
	\item $\mcP(\mathbb{R}^d)$ is a topology
\end{enumerate}
If $\mcT_{S}$ is the standard topology, then
$$\{\emptyset, \mathbb{R}^d\}\subset \mcT_{S}\subset \mcP(\mathbb{R}^d)$$}

\dfn{Coarse and Fine}{Let $X$ be a set, and $\mcT_1,\mcT_2$ be two topologies. If $\mcT_1\subset\mcT_2$, we say $\mcT_1$ is \textbf{coarser} than $\mcT_2$, and $\mcT_2$ is \textbf{finer} than $\mcT_1$}.

\begin{note}
	Given $X$ a set, the set $\{\emptyset, X\}$ is called the \textbf{coarse topology} (it is the coarsest topology). The powerset of $X$, $\mcP(X)$, is called the discrete topology (it is the finest).
\end{note}

\ex{}{Consider $X=\{0,1,2\}$. Then $\mcT = \{\emptyset, X, \{0\}, \{1,2\}\}$ is a topology on $X$. It's easy to check that it's stable under union, stable under finite intersection, and contains both $\emptyset$ and $X$.\\
TODO: input illustration \\
There exist many different topologies on $X$. The number of unique topologies is bounded above by the cardinality of $\mcP(\mcP(X))$, which in this case is $2^{2^3}$ or $256$. Another example of such a topology is $\mcT_2 = \{\emptyset, X, \{0\}\}$. A non-example of a topology is $\{\emptyset, X, \{0\}, \{1\}, \{2\}\}$, since it is not stable under unions.}

\dfn{Neighborhood}{Let $(X,\mcT)$ be a topological space. A subset $A\subset X$ is a neighborhood of $x\in X$ if $\exists U\in X$ open such that $x\in U\subset A$.}

\thm{}{Let $(X,\mcT)$ be a topological space. Then $\forall x\in X, \mcN(x)$ is defined as the collection of all neighborhoods of $x$. That is 
$$\mcN(x) = \{N\subset X| N \text{ is a neighborhood of } x\}$$
Then, $\forall x\in X$ the following statements hold:
\begin{enumerate}
	\item $X\in\mcN(x)$
	\item $\forall N\in\mcN(x),x\in N$
	\item $\forall N\in\mcN(x), \forall A\subset X,$ if $N\subset A,$ then $A\subset \mcN(x)$
	\item $\forall N_1,...,N_k\in \mcN(x), N_1\cap...\cap N_k\in\mcN(x)$
	\item $\forall N\in\mcN(x), \exists N'\in\mcN(x)$ such that $\forall y\in N', N\in\mcN(y)$
\end{enumerate}
Moreover, $\forall U\subset X, U$ is open if and only if $U$ is a neighborhood of all its points.
}

\begin{myproof}
	TODO: Insert proof here
\end{myproof}


\section{Sequences in Topological Spaces}
While the sequences we study are not remarkably different than those encountered in the typical analysis course, we will generalize the concept slightly, especially in regards to the definition of convergence of sequences.

\dfn{Convergence and Continuity}{Let $(X,\mcT)$ be a topological space, $(x_n)_{n}$ in $X$ and $x\in X$. We say that $x_n\to x$ if
$$\forall N\text{ a neighborhood }, \exists n_0\in\mathbb{R} \text{ such that }\forall n > n_0, x_n\in N$$}

\ex{}{Let $X=\{0,1,2\}$ and $\mcT = \{\emptyset, X, \{0\}, \{0,1\}\}$. Then $x_n\to 1$ if and only if $\exists n_0\in\mathbb{R}$ such that $\forall n > n_0, x_n = 0$ or $1$. In particular $0\to 1$.}

\thm{}{Let $(X,\mcT)$ be a topological space. Let $(x_n)_{n\geq 0}$ be a sequence in $X$ that is constant after some time: $\exists n_0\geq 0$ integer such that $x_n=x_{n_0} \forall n\geq n_0$. Then $x_n\to x_{n_0}$ as $n\to\infty$.}
\begin{myproof}
	Let $N$ be a neighborhood of $x_{n_0}$. Then $\forall n\geq n_0, x_n = x_{n_0}$, so $x_n\to x_{n_0}$.
\end{myproof}

\dfn{Continuity}{Let $(X,\mcT)$ and $(Y, \mcT')$ be topological spaces. A map $f:X\to Y$ is called continuous if $$\forall U\subset Y \text{ open, } f^{-1}(U)\subset X \text{ is open}$$}

\thm{}{$f:\mathbb{R}^d\to\mathbb{R}^k$ is continuous if and only if $\forall x\in\mathbb{R}^d,\forall\epsilon > 0,\exists\delta < 0$ such that $f(B(x,\delta))\subset B(f(x),\epsilon)$.}

\begin{myproof}
	$(\implies)$ Assume that $f:\mathbb{R}^d\to\mathbb{R}^k$ is continuous. Let $x\in\mathbb{R}^d$ and $\epsilon>0$. We know $B(f(x),\epsilon)$. We know $B(f(x),\epsilon)$ is open, so $f^{-1}(B(x,\epsilon))$ is open by continuity. Then, since $x\in f^{-1}(B(x,\epsilon)),\exists\delta > 0$ such that $B(x,\delta)\subset f^{-1}(B(f(x),\epsilon))$. Then, $f(B(x,\delta))\subset B(f(x),\epsilon)$. \\ 
	$(\impliedby)$ Assume $\forall x\in\mathbb{R}^d,\forall\epsilon > 0,\exists\delta < 0$ such that $f(B(x,\delta))\subset B(f(x),\epsilon)$. Let $U\subset\mathbb{R}^d$ be open. Then, for $f$ to be continuous we want to show that $f^{-1}(U)$ is open. Let $x\in f^{-1}(U)$. We have that $f(x)\in U$ is open, so $\exists > 0$, such that $B(f(x),\epsilon)\subset U$. By assumption, $\exists\delta>0$ such that $f(B(x,\delta))\subset B(f(x),\epsilon)$. Then $B(x,\delta)\subset f^{-1}(B(f(x),\epsilon))\subset f^{-1}(U)$. Therefore, $f^{-1}(U)$ is open. Therefore, $\forall U\in\mathbb{R}^k$ open, $f^{-1}(U)\subset\mathbb{R}^d$ is open.

	Therefore, $f:\mathbb{R}^d\to\mathbb{R}^k$ is continuous if and only if $\forall x\in\mathbb{R}^d,\forall\epsilon > 0,\exists\delta < 0$ such that $f(B(x,\delta))\subset B(f(x),\epsilon)$.
\end{myproof}


\ex{}{Let $X = \{-1,0,1\}$ and $\mcT = \{\emptyset, X, \{-1\}, \{1\}, \{-1,1\}\}$ be a topology on $X$.

Then $f:\mathbb{R}\to X$ defined by the rule
$$x\mapsto\begin{cases}
	-1 & x < 0 \\ 
	0 & x = 0 \\
	1 & x > 1 \\
\end{cases}$$
Then $f$ is continuous, because
\begin{itemize}
	\item $f^{-1}(\emptyset) = \emptyset$
	\item $f^{-1}(X) = \mathbb{R}$
	\item $f^{-1}(\{-1\}) = (-\infty, 0)$
	\item $f^{-1}(\{1\}) = (0,\infty)$
	\item $f^{-1}(\{-1,1\}) = (-\infty,0)\cup(0,\infty)$
\end{itemize}
are all open.
}

\section{Bases of Topologies}

\dfn{The Subspace Topology}{Let $(X,\mcT)$ be a topological space and $Y\subset X$ be a subset. The subspace topology on $Y$ is
$$\mcT_{|Y} = \{U\cap Y| U\in\mcT\}$$
}

Proof that the subspace topology is, in fact, a topology
\begin{myproof}
	\begin{enumerate}
		\item $\emptyset = \emptyset\cap Y\in\mcT_{|Y}$ and $Y = X\cap Y\subset\mcT_{|Y}$, so $\mcT_{|Y}$ satisfies the first condition for being a topology.
		\item Let $(V_\alpha)_{\alpha\in I}$ be a family of sets in $\mcT_{|Y}$. Then, by the definition of the subspace topology, $\forall \alpha\in I,\exists U_\alpha\in\mcT$ such that $V_\alpha = U_\alpha\cap Y$. Then
		$$\bigcup_{\alpha\in I}V_\alpha = \bigcup_{\alpha\in I}(U_{\alpha}\cap Y) = \bigcup_{\alpha\in I}(U_{\alpha})\cap Y$$
		By the definition of a topology, $\bigcup_{\alpha\in I}(U_{\alpha})\in \mcT$, so $\bigcup_{\alpha\in I}(V_\alpha)\in\mcT_{|Y}$. Therefore, $\mcT_{|Y}$ is stable under unions, satisfying the second condition of being a topology.
		\item Let $V_1,...,V_n\in\mcT_{|Y}$. Then, $\forall i = 1,...,n,\exists U_i\in\mcT$ such that $V_i = U_i\cap Y$. Then $$V_1\cap...\cap V_n = (U_1\cap Y)\cap ...\cap(U_n\cap Y) = (U_1\cap...\cap U_n)\cap Y$$
		Then, clearly, $U_1\cap...\cap U_n\in\mcT$, so $V_1\cap...\cap V_n\in\mcT_{|Y}$. Thus, $\mcT$ is stable under finite intersection, satisfying the third condition of being a topology.
	\end{enumerate}
	Therefore, $\mcT_{|Y}$ is a topology.
\end{myproof}

\ex{}{\begin{enumerate}
	\item $\barB(0,1), S(0,1) = \barB(0,1)\setminus B(0,1) = \{x\in\mathbb{R}^d|\, ||x|| = 1\}$
	\item Torus (same definition as before). TODO: insert definition and illustration
	\item Cantor Set. TODO: insert definition
\end{enumerate}}

\thm{Moore's Law}{Let $X$ be a set and $(\mcT_{\alpha})_{\alpha\in I}$ be a family of topologies on $X$. Then
$$\mcT = \bigcap_{\alpha\in I}\mcT_{\alpha} \subset \mcP(X)$$
is also a topology and we have
$$U\in\mcT \iff U\text{ is open } \forall\mcT_{\alpha}$$}

\begin{myproof}
	TODO: insert proof of Moore's Law here.
\end{myproof}

dfn{Topology Generated by a Subset}{Let $X$ be a set and $A\subset\mcP(X)$. Then the \textbf{topology generated by $A$} is the intersection of all topologies containing $A$. Then $A$ is called a subbasis for this topology.}

\ex{}{The standard topology on $\mathbb{R}^d$ is generated by the set of open balls. (exercise)


TODO: Prove}

\dfn{}{
	Let $X$ be a set. A basis of $X$ is a $\mcB\subset\mcP(X)$ such that
	\begin{enumerate}
		\item $\displaystyle\bigcup_{B\in\mcB} B = X$. In words, $\mcB$ \textit{covers} $X$.
		\item $\forall B_1,B_2\in\mcB, \forall x\in B_1\cap B_2, \exists B_3\in\mcB$ such that $x\in B_3\subset B_1\cap B_2$.
	\end{enumerate}
	Moreover, $\mcB$ is a basis for a topology $\mcT$ if $\mcB\subset\mcT$ and $\mcB$ generates $\mcT$.
}

\ex{}{Open balls in $\mathbb{R}^d$ form a basis for the standard topology. 
\begin{myproof}
	The first condition is self evident. To check that the second condition from the definition of a basis holds:

	Let $x_1,x_2\in\mathbb{R}^d, r_1,r_2\geq 0$. Let $x_3\in B(x_1,r_1)\cap B(x_2, r_2)$. Set $r_3 = \min(r_1 - d(x_1,x_3),r_2 - d(x_2,x_3))$. Then $B(x_3, r_3)\subset B(x_1,r_1)\cap B(x_2,r_2)$.
	Therefore, $\mcB$ is a basis for $\mcT$.
\end{myproof}
}

\begin{note}
	Note that any topology satisfies the conditions for being a basis.
\end{note}

\ex{}{On $\mathbb{R}, \mcB = \{[x,y)|x<y\in\mathbb{R}\}$ is a basis.

\begin{myproof}
	\begin{enumerate}
		\item $\displaystyle\mathbb{R} = \bigcup_{i\in\mathbb{R}} [t,t+1)$
		\item Let $B_1 = [x_1,y_1)$ and $B_2 = [x_2,y_2)\subset\mathbb{R}$. Let $x\in B_1\cap B_2$. Then we have
		$$ \max(x_1,x_2)\leq x < \min(y_1,y_2) $$
		so take $x_3 = \max(x_1,x_2)$, and $y_3 = \min(y_1,y_2)$. Then
		$$ x_3\leq x < y_3$$
		Let $B_3 = [x_3,y_3)$. Then $x\in B_3\subset B_1\cap B_2$.
	\end{enumerate}

	Therefore, $\mcB$ is a basis of $\mathbb{R}$.
\end{myproof}}

\thm{}{
	Let $X$ be a set, and $\mcT\subset\mcP(x)$. Then $\mcT$ is a topology if and only if the following hold:
	\begin{enumerate}
		\item $\emptyset, X\in\mcT$
		\item $\mcT$ is stable under union
		\item $\forall U,V\in\mcT, U\cap V\in\mcT$
	\end{enumerate}
}
To prove the above theorem, the forward direction is tribial. For the reverse direction, the first two conditions are also just assumed, and the proofs of those statements are trivial. For the third condition, we simply need to induct on $n$ to demonstrate stability under finite intersection. Providing a rigorous proof is left as an exercise to the reader.

\thm{}{
	Let $(X,\mcT)$ be a topological space, and let $\mcB\subset\mcT$. Then $\mcB$ is a basis for $\mcT$ if and only if
	$$ \forall U\in\mcT, \forall x\in U, \exists B\in\mcB \text{ such that } x\in B\subset U $$
}

\begin{myproof}
	$(\implies)$ Assume that $\mcB$ is a basis for some $\mcT$.Let $\mcT'$ be the collection of all $U\subset X$ such that $(\forall x\in U, \exists B\in\mcB$ such that $x\in B\subset U)$. Clearly, $\mcB\subset\mcT'$. We then want to show that $\mcT'$ is a topology.
	\begin{enumerate}
		\item $\emptyset\in\mcT'$ and $X\in\mcT'$ since $X = \bigcup_{B\in\mcB}B$
		\item Let $(U_\alpha)_(\alpha\in I)$ be a family in $\mcT'$. Let $x\in U = \bigcup_{\alpha\in I}U_\alpha$.
	\end{enumerate}

	TODO: finish inputting proof
\end{myproof}

\cor[]{}{
	Let $X$ be a set, and $A\subset\mcP(X)$. Let $\mcT$ be the topology generated by $A$. Then
	$$\mcB = \{A_1\cap...\cap A_n | A_i\in A\}\cup\{\emptyset, X\}$$
	is a basis for $\mcT$. Therefore, $U\subset X$ is $\mcT$-open if and only if it is $\emptyset$ or $X$, or if it is a union of finite intersections of elements of $A$.
}

\begin{myproof}
	\begin{enumerate}
		\item $X\in\mcB$
	\end{enumerate}

	TODO: complete proof
\end{myproof}

\section{The Product Topology}
\dfn{The Product Topology}{
	Let $(X_1,\mcT_1)$ and $(X_2,\mcT_2)$ be topological spaces.
	The \textbf{product topology} on $X_1\times X_2$ is the topology generated by
	$$\{U_1\times U_2|U_1\in\mcT_1, U_2\in\mcT_2\}$$
}

\thm{}{Let $(X_1,\mcT_1),(X_2,\mcT_2)$ be topological spaces and let $\mcB_1,\mcB_2$ be bases of $X_1, X_2$ respectively. Then
	$$\mcB = \{B_1\times B_2| B_i\in\mcB_i\}$$
	is a basis for the product topology on $X = X_1\times X_2$. In particular, $\{U_1\times U_2|U_i\in\mcT_i\}$
}

\begin{myproof}
	\begin{enumerate}
		\item $\displaystyle X = X_1\times X_2 = (\bigcup_{B_1\in\mcB_1}B_1)\times (\bigcup_{B_2\in\mcB_2}B_2) = \bigcup_{(B_1,B_2)\in\mcB_1\times\mcB_2}B_1\times B_2 = \bigcup_{B\in\mcB}B$
		\item TODO: finish proof
	\end{enumerate}
\end{myproof}

\ex{}{The product topology $\mathbb{R}\times\mathbb{R}$ is the standard topology. Open balls are not products of open sets of $\mathbb{R}$.}

Let $X$ be a set and $\mcB\subset\mcP(X)$. Consider the set
$$\mcT = \{U\subset X|\forall x\in U,\exists B\in\mcB \text{ such that } x\in B\subset U\}$$
Under what conditions on $\mcB$ is $\mcT$ a topology?

For the first condition, we have $\emptyset\in\mcT$, since $\emptyset$ satisfies the condition. Then, for $X\in\mcT$, we need $\mcB$ to cover $X$.

The second condition is satisfied without further specification, since any union of sets in $\mcT$ inherently contains the elements of $B$ that caused the sets to be in $\mcT$ in the first place.

For the third condition, we only need $\forall U,V\in\mcT, U\cap V\mcT$. For this to hold for this particular $\mcT$, we consider that $x\in U\cap V$ implies that $\exists B_1,B_2$ such that $x\in B_1\subset U$ and $x\in B_2\subset V$. From there, we need the second axiom of the basis for $\mcT$ to be a topology.

\thm{}{Let $X$ be a set, $\mcB$ be a basis, and $\mcT$ the topology generated by $\mcB$. Then $U\subset X$ is $\mcT$-open if
$\forall x\in U, \exists B\in \mcB$ such that $x\in B\subset U$. 

Additionally, $N\subset X$ is a neighborhood of $x$ if $\exists B\in\mcB$ such that $x\in B\subset N$.}

\begin{note}
	Let $X_1,X_2$ be topological spaces. Then $N\subset X_1\times X_2$ is a neighborhood of $X_1\times X_2$ for the product topology if and only if $\exists U_1\subset X_1, U_2\subset X_2$ both open such that $x\in U_1\times U_2\subset N$.
\end{note}

\dfn{Cartesian Product}{
	Let $(X_\alpha)_{\alpha\in I}$ be a family of sets. Then $\prod_{\alpha\in I}X_\alpha$ is the set of lists $(x_\alpha)_{\alpha\in I} = x$ such that for all $\alpha\in I$, $x_\alpha\in X_\alpha$ (the $\alpha$-coordinate of $x$).
}

\ex{}{If $I = \{1,2\}$, then $\prod_{\alpha\in I}X_\alpha = X_1\times X_2$. If $X_\alpha = X \forall \alpha\in I$, then $\prod_{\alpha\in I}X_\alpha = X^{I}$ is identified with the set of functions $f:I\to X$.}

\dfn{The Box Topology}{Let $((X_\alpha,\mcT_\alpha))_{\alpha\in I}$ be a family of topological spaces. The \textbf{Box Topology} on $\prod_{\alpha\in I}X_\alpha$ is the topology generated by the basis
$$\mcB = \{\prod_{\alpha\in I}U_\alpha| U_\alpha\subset X_\alpha\text{ open }\forall\alpha\in I\}$$
}

\cor{}{
	Let $I = \mathbb{N}$ and $X_\alpha = \mathbb{R}, \forall\alpha\in I$. So $$\prod_{\alpha\in I} X_\alpha = \mathbb{R}^{\mathbb{N}}=\{f:\mathbb{N}\to\mathbb{R}\}$$

	Then no sequence of positive functions can converge to $f:\mathbb{N}\to\mathbb{R}$ defined by $k\mapsto 0$. We want to find a neighborhood $N$ of $f$ such that $f_n\not\in N, \forall n\geq 0$.

	Set 
	$$N = (-f_0(0), f_0(0))\times(-f_1(1), f_1(1))\times...(-f_k(k), f_k(k))\times...$$
	Then $f_0(0)\not\in\mathbb{N}$, but $f_k(k)\in\mathbb{N}$.
}

\dfn{General Product Topology}{
	Let $((X_\alpha,\mcT_\alpha))_{\alpha\in I}$ be a family of topological spaces. The \textbf{product topology} on $\prod_{\alpha\in I}X_\alpha$ is the topology generated by the basis
	$$\mcB = \{\prod_{\alpha\in I}U_\alpha|\forall\alpha\in I,U_\alpha\subset \mcT_\alpha\text{ and }\{\alpha\in I|U_\alpha \neq X_\alpha\}\text{ is finite}\}$$
}

\thm{}{Let $(X_\alpha)_{\alpha\in I}$ be a family of topological spaces. Let $\alpha\in I$. Set $\pi_\alpha:\prod_{\beta\in I}X_\beta\to X_\alpha$, called the \textbf{projection map}. Then $\pi_\alpha$ is continuous for the product topology.}

\begin{myproof}
	Let $\alpha\in X_\alpha$ be open. Then, to demonstrate continuity, we need to show that $\displaystyle \pi_{\alpha}^{-1}(U_\alpha)\subset X = \prod_{\beta\in I}X_\beta$ is open. Then, 
	$$\pi_{\alpha}^{-1}(U_\alpha) = \prod_{\beta\in I} U_{\beta}\text{ where } B_\beta = X_\alpha \text{ if }\beta\neq \alpha$$
	$$\pi_{\alpha}^{-1}(U_\alpha) = \prod_{\beta\in I\setminus\{\alpha\}}X_\beta\times U_\beta$$
\end{myproof}

\begin{note}
	The product topology is generated by
	$$\mcA = \{\pi_\alpha^{-1}(U_\alpha)|\alpha\in I, U_\alpha\subset X_\alpha\text{ is open}\}\subset\mcB$$
\end{note}
Indeed, let $B = \prod_{\alpha\in I}U_\alpha\in\mcB$. Then $J = \{\alpha\in I| U_\alpha\neq X_\alpha\}$. Then $B = \bigcap_{\alpha\in J}\pi_{\alpha}^{-1}(U_\alpha)$ the finite intersection of $\mcA$.

\thm{}{
	Let $(X_\alpha)_{\alpha\in I}$ be a family of topological spaces. Let $(x_n)_{n\geq 0}$ be a sequence in $\prod_{\alpha\in I}X_\alpha$. Then $\forall n\geq 0, \pi_\alpha(x_n)$ is the $\alpha$-coordinate of $x_n$. Then 
	$$x_n\to x \iff \forall \alpha\in I, \pi_{\alpha}(x_n)\to\pi_{\alpha}(x) \text{ in } X_\alpha$$
}

\begin{myproof}
	$(\implies)$ Assume $x_n\to x$. Let $\alpha\in I$. Then we want to show that $\pi_{\alpha}(x_n)\to\pi_{\alpha}(x)$. Let $N\subset X_\alpha$ be a neighborhood of $\pi_{\alpha}(x)$. Then, since $\pi_\alpha$ is continuous, we have that $\pi_{\alpha}^{-1}(N)$ is a neighborhood of $x$. Since $x_n\to x$, $\exists n_0$ such that $\forall n > n_0, x_n\in\pi_{\alpha}^{-1}(N)$, by definition. Then, $\forall n>n_0, \pi_{\alpha}(x_n)\in N$. Therefore, $\forall N$ neighborhood of  $\pi_{\alpha}(x), \exists n_0$ such that $\forall n>n_0, \pi_{\alpha}(x_n)\in N$. Thus, $\pi_{\alpha}(x_n)\to\pi_{\alpha}(x)$. Since $\alpha \in I$ is arbitrary, we have $\forall \alpha\in I, \pi_{\alpha}(x_n)\to\pi_{\alpha}(x) \text{ in } X_\alpha$. Therefore,
	$$x_n\to x \implies \forall \alpha\in I, \pi_{\alpha}(x_n)\to\pi_{\alpha}(x) \text{ in } X_\alpha$$

	$(\impliedby)$ Now assume $\forall \alpha\in I, \pi_{\alpha}(x_n)\to\pi_{\alpha}(x) \text{ in } X_\alpha$. Let $N$ be a neighborhood of $x\in X$. Then $\forall\alpha\in I, \exists U_\alpha\subset X_\alpha$ an open set such that $J = \{\alpha\in I|U_\alpha\neq X_\alpha\}$ is finite and $x\in\prod_{\alpha\in I} U_\alpha\subset N$. Then, since $\pi_{\alpha}(x_n)\to \pi_{\alpha}(x)$, we have that  $\forall\alpha\in J, \exists n_\alpha$ such that $\forall n > n_\alpha, \pi_\alpha(x_n)\in U_\alpha$. Set $n_0 = \max\{n_\alpha|\alpha\in J\}$, which is well defined since $J$ is finite. Then $\forall n > n_0, x_n\in \bigcap_{\alpha\in J}\pi_{\alpha}^{-1}(U_\alpha) = \prod_{\alpha\in I}U_{\alpha}\subset N$. So $x_n\to x$.

	Therefore,
	$$x_n\to x \iff \forall \alpha\in I, \pi_{\alpha}(x_n)\to\pi_{\alpha}(x) \text{ in } X_\alpha$$
\end{myproof}

\section{Metric Spaces}
\dfn{Metric Space}{
	Let $X$ be a set. A \textbf{metric} on $X$ is a map $d:X\times X\to\mathbb{R}$ such that $\forall x,y,z\in X$
	\begin{enumerate}
		\item $d$ is symmetric:
		$$d(x,y) = d(y,x)$$
		\item $d$ is positive definite
		$$d(x,y)\geq 0 \text{ with equality if and only if } x=y$$
		\item $d$ satisfies the triangle Inequality
		$$d(x,z)\leq d(x,y) + d(y,z)$$
	\end{enumerate}
	Then the pair $(X,d)$ is called the \textbf{metric space}.
}

\ex{Some examples}{
	The Euclidean metric, or $L^2$ metric on $\mathbb{R}^k$
	$$d(x,y) = \sqrt{|x_1-y_1|^2+...+|x_k-y_k|^2}$$
	$L^1$-metric:
	$$d(x,y) = |x_1-y_1|+...+|x_k-y_k|$$
	$L^{\infty}$-metric:
	$$d(x,y) = \max(|x_1-y_1|,...,|x_k-y_k|)$$
}

\dfn{The Metric Topology}{
	Let $(X,d)$ be a metric space. Then, $\forall x\in X, r\geq 0$ we define the open ball as
	$$B_d(x,r) = \{y\in X|d(x,y) < r\}$$
	The \textbf{metric topology} induced by $d$ is the topoogy generated by the set of open balls, which form a basis.
	Then, $N\subset X$ is a neighborhood of $x$ is $\exists\epsilon > 0$ such that $B_d(x,\epsilon)\subset N$.

	A topological space is \textbf{metrizable} if $\exists d$ a metric which induces it.
}

\thm{}{
	Let $X$ be a set and $d,d'$ two metrics with corresponding metric topologies $\mcT,\mcT'$. Then $\mcT$ is finer than $\mcT'$ if and only if
	$$\forall x\in X, r > 0, \exists\epsilon > 0 \text{ such that } B_d(x,\epsilon)\subset B_{d'}(x,r)$$
}

\begin{myproof}
	$(\implies)$ Assume $\mcT$ is finer than $\mcT'$. Let $x\in X$ and $r > 0$. Then $B_{d'}(x,r)$ is $\mcT'$-open and hence is $\mcT$-open since $\mcT'\subset \mcT$. Then $B_{d'}(x,r)$ is a $\mcT$ neighborhood of $x$, so $\exists \epsilon>0$ such that $B_d(x,\epsilon)\subset B_{d'}(x,r)$.

	$(\impliedby)$ Assume $\forall x\in X, r > 0, \exists \epsilon > 0$ such that $B_d(x,\epsilon)\subset B_{d'}(x,\epsilon)$. Let $U\in\mcT'$. We then want to show that $U\in\mcT$. Take $x\in U$. As $U$ is a $\mcT'$-neighborhood of $x$, we have that $\exists\epsilon>0$ such that $B_{d'}(x,\epsilon)\subset U$. By assumption, $\exists \delta > 0$ such that $B_{d}(x,\delta)\subset B_{d'}(x,\epsilon)\subset U$. Therefore, $U$ is a $\mcT$-neighborhood of $x$, so $U\subset \mcT$.
\end{myproof}

\ex{}{The metrics $d, d_1,$ and $d_\infty$ all induce the same topology. For instance

TODO: Complete example}

\thm{}{
	Let $(X,d_X)$ be a metric space, and $Y$ a topological space. A sequence $(x_n)_{n\geq 0}\subset X$ converges to $x\in X$ if $\forall \epsilon > 0 \exists n_0$ such that $x_n\in B(x,\epsilon) \forall n>n_0$. Then
	$$f:X\to Y\text{ is continuous }\iff \forall x\in X,\forall N\subset Y\text{ neighborhood of } f(x),\exists \epsilon > 0 \text{ such that } f(B(x,\epsilon))\subset N$$
}

\thm{}{
	Let $X,Y$ be topological spaces, and let $f:X\to Y$ be a mapping.
	\begin{enumerate}
		\item If $f$ is continuous then $\forall (x_n)_{n\geq 0}\subset X$ converging to $x$, we have $f(x_n)\to f(x)$.
		\item Suppose $X$ is metrizable. If $x_n\to x$ implies $f(x_n)\to f(x)$, then $f$ is continuous.
	\end{enumerate}
}

\begin{myproof}
	\begin{enumerate}
		\item Assume $f$ is continuous. Let $x_n\to x\in X$. Then, we want to show that $f(x_n)\to f(x)$. Let $N\subset Y$ be a neighborhood of $f(x)$. Then, since $f$ is continuous, $f^{-1}(N)$ is a neighborhood of $x$ ($\exists f(x)\in U\subset N, f^{-1}(U)$ is open and a subset of $f^{-1}(N)$). Since $x_n\to x, \exists n_0$ such that $\forall n > n_0, x_n\in f^{-1}(N)$. Then $f(x_n)\in N$.
		\item Let $X$ be metrizable, and assume $x_n\to x$ implies $f(x_n)\to f(x)$. Let $d$ be a metric on $X$ inducing the topology. Let $x\in X$ and $N\subset Y$ be a neighborhood of $f(x)$. Suppose for contradiction that $\forall\epsilon>0, f(B(x,\epsilon))\not\subset N$. In particular, $\forall n\geq 1, \exists x_n\in B(x,\frac{1}{n})$ such that $f(x_n)\not\in N$. Now apply our supposition ot $\epsilon=\frac{1}{n}$. Then we can check that $f(x_n)\to f(x)$ and $x_n\to x$ contradicts the statement that $\exists n\geq 1, \exists x_n\in B(x,\frac{1}{n})$ such that $f(x_n)\not\in N$.
	\end{enumerate}
\end{myproof}

\dfn{Homeomorphism}{
	Let $X,Y$ be a topological space. A homeomorphism from $X$ to $Y$ is a bijection $f:X\to Y$ such that $U\subset X$ is open if and only if $f(U)\subset Y$.
	\\
	$X$ and $Y$ are homeomorphic if $\exists$ a homeomorphism.
}

\cor{(of Theorem 1.6.3)}{Let $X,Y$ be topological spaces. Then $f:X\to Y$ is a homeomorphism if and only if $f$ is a bijection such that $x_n\to x$ if and only if $f(x_n)\to f(x)$.}

\dfn{Boundary Metric of $d$}{Let $(X, d)$ be a metric space. Then $\bard(x,y) = \min(d(x,y), 1)\leq 1$ is a metric that defines the same topology.}

\begin{myproof}
	Symmetry and positive definitneness are trivial.

	To prove the triangle inequality:

	Let $x,y,z\in X$. Case 1: assume $d(x,y)\geq 1$ of $d(y,z)\geq 1$. Then $\bard(x,y)+\bard(y,z)\geq 1\leq \bard(x,z)$.

	Case 2: $d(x,y) = \bard(x,y)$ and $d(y,z) = \bard(y,z)$. Then
	$$\bard(x,y) + \bard(y,z) = d(x,y) + d(y,z)\geq d(x,z)\geq \bard (x,z)$$
	Therefore, $\bard$ satisfies the triangle inequality.
\end{myproof}

\dfn{Diameter}{
	Let $(X,d)$ be a metric space, and $A\subset X$. Then we define diameter as
	$$\text{diam}_d(A)=\sup\{d(x,y)|x,y\in A\}$$
}

\ex{}{$\text{diam}_{\mathbb{R}}((0,1)\cup\{2\}) = 2$}

\thm{}{
	Let $((X_\alpha,d_\alpha))_{\alpha\in I}$ be a family of metric spaces. Suppose $\exists c>0$ such that $\forall\alpha\in I, \text{diam}_{d_\alpha}X_\alpha\leq c$. Then
	$$d((x_\alpha)_{\alpha\in I}, (y_\alpha)_{\alpha\in I}) = \sup\{d_\alpha(x_\alpha,y_\alpha)|\alpha\in I\}$$
	is well defined and is a metric (where $\displaystyle((x_\alpha)_{\alpha\in I},(y_\alpha)_{\alpha\in I})\in\prod_{\alpha\in I}X_\alpha$). Moreover, the metric topology is finer than the product topology. We have equality if and only if $\forall\epsilon > 0, I_\epsilon=\{\alpha\in I|\text{diam}X_\alpha\geq \epsilon\}$ is finite.
}

\begin{myproof}
	\begin{enumerate}
		\item First, we must prove that $d$ is a metric. Proving that $d$ is symmetric and positive definite is trivial, so we must demonstrate that $d$ satisfies the triangle inequality.
		
		Let $(x_\alpha)_{\alpha\in I}, (y_\alpha)_{\alpha\in I}, (z_\alpha)_{\alpha\in I}$ be sequences such that $x_\alpha, y_\alpha, z_\alpha\in X_\alpha, \forall\alpha\in I$. Then we want to show that
		$$d((x_\alpha)_{\alpha\in I}, (z_\alpha)_{\alpha\in I})  + d((z_\alpha)_{\alpha\in I}, (y_\alpha)_{\alpha\in I}) \leq d((x_\alpha)_{\alpha\in I}, (y_\alpha)_{\alpha\in I})$$
		For convenience, let $x = (x_\alpha)_{\alpha\in I}, y = (y_\alpha)_{\alpha\in I}$, and $z = (z_\alpha)_{\alpha\in I}$. We have that $\forall \alpha\in I$, 
		$$d(x,z) + d(z,y)\geq d_\alpha(x_\alpha, z_\alpha) + d_\alpha(z_\alpha, y_\alpha)\geq d_\alpha(x_\alpha, y_\alpha)$$
		
		So $d(x,z) + d(z,y)$ is an upper bound of $\{d_\alpha(x_\alpha,y_\alpha|\alpha\in I)\}$. Then $d(x,y)$ is the least upper bound of this set, so
		$$d(x,y)\leq d(x,z) + d(z,y)$$
		Therefore $d$ is a metric and is well defined.

		\item Next, to demonstrate the ``moreover..'' statement, we must demonstrate that the metric topology, $\mcT$, is finer than the product topology, $\mcT'$. That is, we must show that $\mcT'\subset\mcT$.
		
		We have that a basis for $\mcT$ is the set $\mcB = \{\text{open balls}\}$, and a basis for $\mcT'$ is $\mcB' = \{\prod_{\alpha\in I}B(x_\alpha, r_\alpha)| r_\alpha = \infty\text{ for all but finitely many }\alpha\}$. Let $B' = \prod_{\alpha\in I}B(x_\alpha, r_\alpha)\in\mcB'$. Let $(y_\alpha)_{\alpha\in I}\in \mcB'$. Set $\epsilon = \inf\{r_\alpha - d_\alpha(x_\alpha, y_\alpha)\}$. Note that this is necessarily greater than 0, and equal to $\infty$ for ``most'' $\alpha$'s. Then 
		$$B((y_\alpha)_{\alpha\in I}) = \prod_{\alpha\in I}B(y_\alpha, \epsilon)\subset B(x_\alpha, r_\alpha)\subset B'$$
		So $B'$ is a neighborhood of $(y_\alpha)_{\alpha\in I}$. Therefore $B'\in \mcT$, so $B'\in\mcT'$. Hence, the metric topology $\mcT$ is finer than $\mcT'$.

		\item Suppose $I_\epsilon = \{\alpha\in I|\text{diam}X_\alpha\geq\epsilon\}$ is finite $\forall\epsilon > 0$. We want to show that $\mcT\subset\mcT'$. Let $x = (x_\alpha)_{\alpha\in I}\in X$ and $r > 0$. Then, let $y\in B(x,r)$. Set $\epsilon = r - d(x,y)$. Then $B(y,\epsilon)\subset B(x,r)$. Then we want to find $N$, a $\mcT'$-neighborhood of $y$ contained in $B(y,\epsilon)$. So, $\forall\alpha\in I_{\epsilon/2}$, set $\epsilon_\alpha = \frac{\epsilon}{2}$. Then, $\forall \alpha\in I\setminus I_{\epsilon / 2}$, set $\epsilon_\alpha = \infty$. Then $\prod_{\alpha\in I}B(y_\alpha,\epsilon)\subset B(y,\epsilon)$, and $\prod_{\alpha\in I}B(y_\alpha,\epsilon)$ is a $\mcT'$-neighborhood $y$. Indeed, if $(z_\alpha)_{\alpha\in I}\in\prod_{\alpha\in I}B(y_\alpha,\epsilon_\alpha)$. Then $\forall\alpha\in I$, if $\alpha\in I_{\epsilon}$, then $d_{\alpha}(y_\alpha,z_\alpha) \leq \epsilon/2$. If $\alpha\in I \setminus I_\epsilon, d_{\alpha}(y_\alpha,z_\alpha)\leq \epsilon/2$ as $\text{diam}X_\alpha\leq \epsilon/2$.
		
		\item TODO: finish proof later
	\end{enumerate}
\end{myproof}

\thm{}{
	Let $(X,d)$ be a metric space. Then
	$$x_n\to x\text{ in the metric topology} \iff d(x_n, x)\to 0 \text{ in the standard topology of }\mathbb{R}$$
}

\ex{}{
	In $\mathbb{R}^{\mathbb{Z}\geq 0} = \{f:\mathbb{Z}_{\geq 1}\to\mathbb{R}\} = \{(x_n)_{n\geq 1}\text{ sequence in }\mathbb{R}\}$. Then $$d((x_n)_{n\geq 1}, (y_n)_{n\geq 1}) = \sup\{\min(|x_n - y_n|, 1)|n\geq 1\}$$ is a metric. It does not induce the product topology, but rather the \textbf{uniform topology}.

	This works for $\mathbb{R}^I$, where $I$ is any set. For instance, $\mathbb{R}^{\mathbb{R}} = \{f:\mathbb{R}\to\mathbb{R}\}$. Then 
	$$d(f,g) = \sup\{\min(|f(x)-g(x)|, 1)|x\in\mathbb{R}\}$$

	However, $d'((x_n)_{n\geq 1}, (y_n)_{n\geq 1}) = \sup\{\min(|x_n-y_n|, \frac{1}{n})|n\geq 1\}$ induces the product topology.
}

\begin{note}
	In the space of functions from $\mathbb{R}\to\mathbb{R}$, the ball of radius $\epsilon$ centered at a function $f$, $B(f,\epsilon)$, is the set of functions whose graph is contained in the ``$\epsilon$-tubular neighborhood'' of $f$.
\end{note}

\section{The Quotient Topology}
\dfn{Inclusion Map}{
	Let $X$ be a topological space, and $Y\subset X$. We call $i:Y\to X$ defined by the rule $y\mapsto y$. Then the subspace topology on $Y = \{U\cap Y| U\subset X\}$ where $U\cap Y = i^{-1}(U)$. So the subspace topology is $\{i^{-1}(U)| U\subset X \text{ is open}\}$. This is the coarsest topology that makes $i$ continuous.
}

\dfn{}{
	Let $X$ be a topological space, $Y$ a set, and $f:X\to Y$ a map. Then $\mcT = \{U\cap Y| f^{-1}(U)\subset X\text{ is open}\}$ is a topology called the \textbf{Quotient Topology}. It is the finest topology that makes $f$ continuous.
}

\begin{myproof}
	Let $X$ be a topological space and $\mcT = \{U\cap Y| f^{-1}(U)\subset X\text{ is open}\}$. Then we want to demonstrate that $\mcT$ is indeed a topology.
	\begin{enumerate}
		\item $f^{-1}(U) = \emptyset$, which is open, so $\emptyset\in\mcT$. Then $f^{-1}(Y) = X$, which is open. Therefore, $Y\in\mcT$.
		\item Let $(U_\alpha)_{\alpha\in I}$ be a family of sets in $\mcT$. Then 
		$$f^{-1}(\bigcup_{\alpha\in I}U_\alpha) = \bigcup_{\alpha\in I}f^{-1}(U_\alpha)$$
		By definition of $\mcT$, $f^{-1}(U_\alpha)$ is open in $X$ for all $\alpha\in I$. Therefore, it's union is open since openness is stable under union. So $f^{-1}(\bigcup_{\alpha\in I}U_\alpha)$ is open in $X$. Then, by the definition of $\mcT$, we have $\bigcup_{\alpha\in I}U_\alpha\in\mcT$. Thus, $\mcT$ is stable under union.
		\item Let $U,V\in\mcT$. We want to demonstrate that $U\cap V\in\mcT$. We have that 
		$$f^{-1}(U\cap V) = f^{-1}(U)\cap f^{-1}(V)$$
		Note that $f^{-1}(U)$ and $f^{-1}(V)$ are both open in $X$, so their intersection is also open in $X$. Therefore, by definition of $\mcT$, we have $U\cap V\in \mcT$. Thus, $\mcT$ is stable under finite intersection.
	\end{enumerate}
	$\mcT$ is therefore a topology.
\end{myproof}

\ex{}{Let $f:\mathbb{R}^2\to\mathbb{R}$ be defined by $(x,y)\mapsto x$. Then the quotient of the standard topology on $\mathbb{R}^2$ under $f$ is the standard topology on $\mathbb{R}$. The proof is left as an exercise.}

\dfn{Equivalence Relation}{
	Let $X$ be a set. A \textbf{relation} is a map $\sim:X\times X\to \{\text{True, False}\}$, where $\text{True}$ indicates that $x$ and $y$ are in relation, and $\text{False}$ indicates that $x$ and $y$ are not in relation.

	The map $\sim$ is called an $\textbf{equivalence relation}$ if it satisfies the following conditions $\forall x,y,z\in X$:
	\begin{enumerate}
		\item \underline{\textbf{Symmetry:}} $x \sim y\iff y\sim x$
		\item \underline{\textbf{Reflexivity:}} $x\sim x$
		\item \underline{\textbf{Transitivity:}} $(x\sim y\text{ and } y\sim z)\implies x\sim z$
	\end{enumerate}
	If $\sim$ is an equivalence relation, the \textbf{equivalence class} of $x\in X$ is $\barx = \{y\in X| y\sim x\}$.

	The \textbf{quotient set} is $X/\sim = \{\barx|x\in X\}\subset\mcP(X)$ the set of equivalence classes.

	The \textbf{quotient map} $f:X\to X/\sim$ defined by $x\mapsto\barx$ is surjective.
}

\ex{}{
	Let $X,Y$ be sets, and $f:X\to Y$ be a surjective map. Let $x\sim x'$ if and only if $f(x) = f(x')$. Then $\sim$ is an equivalence relation. Moreover, $\barf:X/\sim\to Y$ defined by $\barx\mapsto f(x)$ is well-defined and bijective.}

\ex{}{
	Let $X\subset [0,1]^2\subset\mathbb{R}^2$ have the subspace topology. Then we want to "identify" $(0,y)$ with $(1,y)$ and $(x,0)$ with $(x,1)$. Set $(x,y)\sim (x', y')$ if and only if
	$$\begin{cases}
		x = x' \text{ and } y = y' \\ 
		x = x' \text{ and } \begin{cases}y=0 \text{ and } y' = 1 \\  y=1 \text{ and } y' = 0\end{cases} \\ 
		y = y' \text{ and } \begin{cases}x=0 \text{ and } x' = 1 \\ x=1 \text{ and } x' = 0
		\end{cases}
	\end{cases}$$
	Check that it is an equivalence relation as an exercise.
	}

\section{Closed Sets and Limit Points}
\dfn{Closed Set}{Let $X$ be a topological space. A subset $C\subset X$ is \textbf{closed} if its complement, $X\setminus C$, is open.}

\ex{}{In $\mathbb{R}$, $\{x\}$ is closed (since $\mathbb{R}\setminus C$ is open). Also, $[a,b]$ is closed, as $(-\infty, a)\cup(b,\infty)$ is open. The sets $(a,b)$ and $[a,b)$ are not closed.
}

\thm{Properties of Closed Sets}{
	Let $X$ be a topological space. Then
	\begin{enumerate}
		\item $\emptyset, X$ are closed.
		\item Closed is stable under intersection.
		\item Closed is stable under finite unions.
	\end{enumerate}
}

\begin{myproof}
	\begin{enumerate}
		\item TODO: Later
	\end{enumerate}
\end{myproof}

\thm{}{Let $X,Y$ be a topological space, and let $f:X\to Y$ be a mapping. Then $f$ is continuous if and only if $\forall C\subset Y, f^{-1}(C)$ is closed in $X$}

\begin{myproof}
	TODO: finish later
\end{myproof}

\ex{}{In $\mathbb{R}^d$, let $f:\mathbb{R}^d\to\mathbb{R}$ be a map. Let $x\in\mathbb{R}^d$, and define $f$ by the rule $y\mapsto d(x,y)$ is continuous. The proof is left as an exercise}

\ex{}{$\{(x,y)\in\mathbb{R}^2| x+y=1\}\subset \mathbb{R}^2$ is closed.}

\thm{}{Let $X$ be a topological space and $C\subset X$. Then
	\begin{enumerate}
		\item if $C$ is closed, then $\forall(x_n)_{n\geq 0}\subset C$ such that $x_n\to x, x\in C$. In this case, we say $C$ is "closed under sequential limits.
		\item Suppose $X$ is metrizable. If $C$ is stable under limit then it is closed.
	\end{enumerate}
}

\begin{myproof}
	\begin{enumerate}
		\item Let $C\subset X$ be closed, and let $(x_n)_{n\geq 0}\subset C$ such that $x_n\to x$ for some $x\in X$. Suppose for contradiction that $x\in X\setminus C$. Since $C$ is closed $X\setminus C$ is open, so it is a neighborhood of $x$. Since $x_n\to x, \exists n_0$ such that $\forall n\geq n_0, x_n\in X\setminus C$. However, this contradicts the fact that $x_n\in C, \forall n\geq 0$. Therefore, $x\in C$, so $C$ is closed under sequential limits.
		\item Let $d$ be a metric inducing a topology on $X$. To attempt a proof by contrapositive, suppose $C$ is not closed, and therefore $X\setminus C$ is not open. Therefore, $\exists x\in X\setminus C$ such that $X\setminus C$ is not a neighborhood of $x$. Then, $\forall \epsilon > 0, B(x,\epsilon)\not\subset X\setminus C$. Then, $\forall n\geq 1, \exists x_n \in B(x,\frac{1}{n})$ such that $x_n X\setminus C$. Therefore, $x_n\in C$. Then $d(x_n,x)<\frac{1}{n}$, so $x_n\to x\not\in C$. Therefore, $C$ is not stable under sequential limits.
	\end{enumerate}
\end{myproof}

\dfn{Closure, Interior, Boundary, and Limit Point}{
	Let $X$ be a topological space and $A\subset X$.
	\begin{enumerate}
		\item The \underline{\textbf{closure}} of $A$, denoted $\barA$, is defined as the intersection of all closed sets containing $A$. This is equivalent to the smallest closed set containing $A$, or the set $\{x\in X|\text{any neighborhood meets } A\}$.
		\item The \underline{\textbf{interior}} of $A$, denoted $\mathring{A}$ or $\text{Int}(A)$, is defined as the union of all open sets contained in $A$. This is equivalent to the biggest set contained in $A$, or the set $\{x\in A| x\text{ has a neighborhood contained in } A\}$.
		\item The \underline{\textbf{boundary}} of $A$, denoted $\partial A$, is defined as $\barA\setminus\mathring{A} = \barA\cap(X\setminus \mathring{A})$. Note that $\partial A$ is closed, and that $\mathring{A}\sqcup\partial A$.
		\item Let $x\in X$. Then $x$ is \underline{\textbf{accumulation}} or \underline{\textbf{limit point}} if all $x$ meets $A\setminus\{x\}$.
	\end{enumerate}
}

Proof of the equivalence of the definition of interior with the listed set
\begin{myproof}
	Let $U = \{x\in A| \exists N\subset A\text{ a neighborhood of }x\}$. We want to show that $U=\mathring{A}$.

	$(\subset)$ To demonstrate that $U\subset\mathring{A}$, we just need to show that $U$ is open. Let $x\in U$. Then $\exists N$ an open neighborhood of $x$ such that $N\subset A$. Then $\exists V\subset X$ open such that $x\in V\subset N\subset A$. Then $V\subset U$. Indeed $\forall y\in V, y\in V\subset A$, so $y\in U$, hence $U$ is a neighborhood of $x$, so $U$ is open. Therefore, $U\subset\mathring{A}$.

	$(\supset)$ Let $x\in\mathring{A}$. Then $x\in\mathring{A}\subset A$. Note that $\mathring{A}$ is a neighborhood of $x$, so $x\in U$. Therefore, $\mathring{A}\subset U$ and $\mathring{A} = U$.
\end{myproof}


\thm{}{
	Let $X$ be a topological space, and $A\subset X$. Then $\overline{X\setminus A} = X\setminus\mathring{A}$ and $\text{Int}(X\setminus A) = X\setminus\barA$.
}

\begin{myproof}
	Let $\mcT = \{C\subset X| C \text{ is closed and contains } X\setminus A\}$. Then $C\in\mcT$ if and only if $X\setminus C$ open and contained in $A$. Let $\mcU = \{U\subset X| U\text{ is open abd contained in } A\}$. Then 
	$$\overline{X\setminus A} = \bigcap_{C\in\mcT}C = \bigcap_{U\in\mcU}X\setminus U = X\setminus(\bigcup_{U\in\mcU}U) = X\setminus \text{Int}(A)$$
\end{myproof}

\thm{}{
	Let $X$ be a topological space and $A\subset X$.Then
	\begin{enumerate}
		\item $\barA\supset\{x\in X| \exists (x_n)_{n\geq 0}\subset A, \text{ such that } x_n\to x\}$
		\item Suppose $X$ is metrizable, then $\bar{A} =  \{x\in X| \exists (x_n)_{n\geq 0}\subset A, \text{ such that } x_n\to x\}$
	\end{enumerate}
}

\ex{}{
	Let $X$ be a topological space and $Y\subset X$ with subspace topology. Then 
	$$\{\text{closed subsets of $Y$}\} = \{C\cap Y| C\subset X\text{ closed}\}$$
	Indeed, if $C\subset X$ closed, then $Y\setminus(C\cap Y) = (X\setminus C)\cap Y$ open in $Y$. If $C'\subset Y$ is closed, then $Y\setminus C'\subset Y$ is open. Then, $\exists U\subset X$ such that $Y\setminus C' = U\cap Y$. Then
	$$C' = Y\setminus (U\cap Y) = (X\setminus U)\cap Y$$
	Note that $X\setminus U$ is closed in $X$.
}

\thm{}{
	Let $A\subset Y$ be a subset. Then $\overline{A}^Y$ is the closure of $A$ in $Y$, and $\overline{A}^Y = \overline{A}^X\cap Y$.

	$\barA^X\cap Y$ is closed and contains $A$, so $\barA^X\cap Y\supset \barA^Y$. Conversely, $\barA^Y = C\cap Y$ for some $C\subset X$ closed. Then $A\subset C$, so $\barA^X\subset C$, and $\barA^X\cap Y\subset C\cap Y$. So $\barA^Y = \barA^X\cap Y$.
}

\cor{}{
	Let $X_1,X_2$ be two topological spaces. Let $C_1\subset X_1$ and $C_2\subset X_2$ be closed. Then $C_1\times C_2$ is closed in $X_1\times X_2$. Indeed,
	$$(X_1 \times X_2)\setminus(C_1\times C_2) = (X_1\times(X_2\setminus C_2))\cup((X_1\setminus C_1)\times C_2)$$
	Note that $X_1\setminus C_1,X_2\setminus C_2$ are open since $C_1,C_2$ are closed. Then, $X_1\times(X_2\setminus C_2)$ and $(X_1\setminus C_1)\times C_2$ are both open, so their union is open.
}

\begin{myproof}
	Indeed, $$(X_1 \times X_2)\setminus(C_1\times C_2) = (X_1\times(X_2\setminus C_2))\cup((X_1\setminus C_1)\times C_2)$$
	Note that $X_1\setminus C_1,X_2\setminus C_2$ are open since $C_1,C_2$ are closed. Then, $X_1\times(X_2\setminus C_2)$ and $(X_1\setminus C_1)\times C_2$ are both open, so their union is open.
\end{myproof}

\begin{note}
	Let $A_1\subset X_1$ and $A_2\subset X_2$. Then $\overline{A_1\times A_2} = \barA_1\times\barA_2$, which is closed and contains $A_1\times A_2$.
	\begin{myproof}
		We wat to show that $\barA_1\times\barA_2\subset \overline{A_1\times A_2}$.

		Let $(x_1,x_2)\in\barA_1\times\barA_2$. Let $N$ be a neighborhood of $(x_1, x_2)$ in $X_1\times X_2$. Then, $\exists U_1\subset X_1$ neighborhood of $x_1$ and $\exists U_2\subset X_2$ neighborhood of $x_2$ such that $U_1\times U_2\subset N$.

		Since, for $i\in \{1,2\},\exists y_i\in U_i\cap A_i$,we have that $(y_1, y_2)\in (U_1\times U_2)\cap(A_1\times A_2)\subset N\cap (A_1\times A_2)$. Therefore, $(x_1,x_2)\in \overline{A_1\times A_2}$
	\end{myproof}
\end{note}

\begin{note}
	For $A,B$ two sets,
	$$\overline{A\cup B} = \barA\cup\barB$$
	\begin{myproof}
		Let $A,B$ be two sets.
		$(\subset)$ Since closed-ness is stable under finite union, we have that $\barA\subset\barB$ is closed and contains $A\cup B$. Therefore, since the closeure of a set is the smallest closed set containing that set, we have that $$\overline{A\cup B}\subset \barA\cup\barB$$
		$(\supset)$ Clearly $A\subset A\cup B$, so $\barA\subset\overline{A\cup B}$. Similarly, $B\subset A\cup B$, so $\barB\subset\overline{A\cup B}$. Therefore, 
		$$\overline{A\cup B}\supset \barA\cup\barB$$
		Thus, $$\overline{A\cup B} = \barA\cup\barB$$
	\end{myproof}
\end{note}

\dfn{Hausdorff Space}{
	Let $X$ be a topological space. Then $X$ is \underline{\textbf{Hausdorff}} if $\forall x\neq y\in X, \exists U$ a neighborhood of $x$ and $\exists V$ a neighborhood of $y$ such that $U\cap V=\emptyset$.
}
Most topological spaces we will look at will be Hausdorff with the exception of some finite ones.

\ex{}{
	\begin{enumerate}
		\item Metric spaces $(X,d)$ are Hausdorff. For $x\neq y\in X$, we only have to look at $B(x,\frac{d(x,y)}{2})\cap B(y,\frac{d(x,y)}{2})=\emptyset$ to see this.
		\item $X = \{0,1,2\}, \mcT\{\emptyset, X, \{0\}, \{1,2\}\}$ is \textbf{not} Hausdorff.
		\item If $X$ is Hausdorff and $Y\subset X$ is a subspace, then $Y$ is Hausdorff. The proof is left as an exercise.
		\item Let $X_1, X_2$ be two Hausdorff spaces. Then $X_1\times X_2$ is Hausdorff. Let $(x_1,x_2)\neq (y_1,y_2)\in X_1\times X_2$. If $x_1\neq y_1$, then $\exists U_1, V_1$ respective neighborhoods of $x,y$ such that $U_1\times V_1= \emptyset$. Then $(U_1\times X_2)\cap (V_1\times X_2) = \emptyset$. The case where $x_1 = y_1$ but $x_2\neq y_2$ is identical.
	\end{enumerate}
}

\thm{}{Let $X$ be a Hausdorff topological space. If $x_n\to x$ and $x_n\to y$ as $n\to\infty$, then $x=y$}
\begin{myproof}
	Assume for contradiction that $x\neq y$. Then $\exists U, V$ respective neighborhoods of $x,y$ such that $U\cap V\neq 0$. Since $x_n\to x, \exists n_0$ such that $\forall n > n_0, x_n\in U$, and since $x_n\to y, \exists k_0$ such that $\forall n > k_0, x_n\in V$. Then, $\forall n > \max(n_0, k_0), x_n\in U\cap V$, which is a contradiction.
\end{myproof}

\thm{}{Let $X$ be a Hausdorff topological space and let $A = \{x_1,...,x_n\}\subset X$ be a finite subset. Then $A$ is closed.}

\begin{myproof}
	TODO: transcribe proof later
\end{myproof}

\begin{note}
	Consider the set of natural numbers $\mathbb{N} = \{0,1,2,...\}$ and take $\mcT = \{U\subset \mathbb{N}|\mathbb{N}\setminus U\text{ is finite}\}$. Then every finite set is closed, but $(\mathbb{N},\mcT)$ is not Hausdorff. The proof is left as an exercise.
\end{note}

\dfn{Continuous at a Point}{
	Let $X,Y$ be topological spaces, take $x\in X$, and let $f:X\to Y$ be a map. Then $f$ is \underline{\textbf{continuous at $x$}} if $\forall N$ neighborhood of $f(x)$, $f^{-1}(N)$ is a neighborhood of $x$.
}

\thm{}{
	Let $X, Y$ be topological spaces. Then $f:X\to Y$ is continuous if and only if it is continuous at every point
}
\begin{myproof}
	TODO: transcribe and complete proof
\end{myproof}

\ex{}{
	Let $X,Y,Z$ be topological spaces.
	\begin{enumerate}
		\item Constant maps are continuous.
		\item Let $A\subset X$. Then the inclusion map $A\hookrightarrow X$ is continuous.
		\item Let $f:X\to Y$ and $g:Y\to Z$ be maps such that $f$ is continuous at $x\in X$ and $g$ is continuous at $f(x)\in Y$. Then $g\circ f:X\to Z$ is continuous at $x$.
		\item Let $A\subset X$ be a subspace and $f:X\to Y$ be a continuous map. Then $f_{|A}:A\to Y$ defined by $a\mapsto f(a)$ is continuous.
	\end{enumerate}

	TODO: finish inserting examples later
}