\chapter{Introduction}

\section{The Standard Topology on Euclidean Space}
Topology, from the greek \textit{topos}, meaning "place" or "locality", and \textit{logos} meaning "study", can be thought of as the study of shape. More specifically, the study of how geometric objects behave under continuous deformations.

There are a variety of different (equivalent) approaches to topology, including but not limited open sets, neighborhoods, metrics, convergence of sequences, and continuity of functions. All of the preceding are discussed in this course, but we will rely heavily on the concept of open sets. Before getting to the subject, we review some important fundamentals.

Firstly, we denote the set of real numbers as $\mathbb{R}$, and the set of $d$-tuples as $\mathbb{R}^{d}$. The latter of these sets is sometimes referred to as "Euclidean $d$-space".

\dfn{The (Standard) Inner Product and (Standard) Norm}{Let $x = (x_1,...,x_d), y=(y_1,...,y_d)\in\mathbb{R}$. The standard inner product is a map $f:\mathbb{R}^d\times\mathbb{R}^d\to\mathbb{R}$ defined by
$$f(x,y) = \langle x,y\rangle = x_1y_1+...x_dy_d$$
We then define the (standard) norm as a function $g:\mathbb{R}^d\to\mathbb{R}$ as 
$$g(x) = ||x|| = \sqrt{\langle x,x\rangle} = \sqrt{x_1^2+...x_d^2}$$}
\thm{}{For all $x,x',y,y'\in\mathbb{R}^d$ the following properties regarding the inner product hold
\begin{enumerate}
	\item Bilinearity:
	$$\langle\lambda x+\lambda'x', \mu y+ \mu'y'\rangle = \lambda\mu\langle x,y\rangle + \lambda'\mu\langle x'y\rangle + \lambda\mu' \langle x,y'\rangle + \lambda'\mu'\langle x',y'\rangle$$
	\item Symmetry:
	$$\langle x, y\rangle = \langle y,x\rangle$$
	\item Positivity:
	$$\langle x,x\rangle \geq 0 \text{ with equality if and only if } x=0$$
	\item Cauchy Schwarz Inequality:
	$$|\langle x,y\rangle|\leq ||x|| \,||y||, \text{ with equality if and only if } x \parallel y$$
	\item Triangle Inequality:
	$$||x+y||\leq ||x|| + ||y||, \text{ with equality if and only } y=0 \text{ or } \exists a\geq 0 \text{ s.t. } x=ay$$
\end{enumerate}}

\begin{myproof}
	TODO: prove the above statements from definition
\end{myproof}



\dfn{The (Standard) Metric on $\mathbb{R}^d$}{The (standard) metric on $\mathbb{R}^d$ is a map $d:\mathbb{R}^d\times\mathbb{R}^d\to\mathbb{R}$ such that $\forall x,y\in\mathbb{R}$
$$d(x,y) = ||x-y||$$}

\thm{}{For all $x,y,z\in\mathbb{R}$, the following properties regarding the (standard) metric hold:
\begin{enumerate}
	\item Positivity:
	$$d(x,y)\geq 0\text{ with equality if and only if }x=y$$
	\item Symmetry:
	$$d(x,y) = d(y,x)$$
	\item Triangle Inequality
	$$d(x,z)\leq d(x,y)+d(y,z)$$
\end{enumerate}}
\begin{myproof}
TODO: prove above statements from definition.
\end{myproof}

\dfn{(Standard) Open Ball and Closed Ball}{Let $x\geq 0$ and $R\geq0$. Then the (standard) open ball is the set
$$B(x, R) = \{y\in\mathbb{R}^d| d(x,y)<R\}$$
and the (standard) close ball is the set
$$\barB(x,R) = \{y\in\mathbb{R}^d| d(x,y)\leq R\}$$}
In $\mathbb{R}$, $B(x,R)$ is just the open interval $(x-R, x+R)$. In $\mathbb{R}^2$, $B(x,R)$ is the interior of the circle centered at $x$ of radius $R$. In $\mathbb{R}^3$, $B(x,R)$ is the the interior of the sphere centered at $x$ of radius $R$. The closed counterparts of these sets include the boundaries of the described sets.

\dfn{Neighborhoods in $\mathbb{R}^d$}{Let $U\subset\mathbb{R}^d$ and $x\in U$. Then $U$ is a \textbf{neighborhood} of $x$ if it contains a nonempty ball centered at $x$, i.e. if
$$\exists\epsilon > 0, B(x,\epsilon)\subset U$$
Additionally, $U$ is a (standard) open ball of $\mathbb{R}^d$ if it is a neighborhood of all its points, i.e.
$$\forall y\in U,\exists \epsilon > 0, B(x,\epsilon)\subset U$$}

\begin{note}
	\begin{enumerate}
		\item $\empty$ and $\mathbb{R}$ are open.
		\item Open balls are open.
	\end{enumerate}
\end{note}

\begin{proof}
	TODO: prove the statements in the note
\end{proof}

\thm{}{
	\begin{enumerate}
		\item Let $(V_\alpha)_{\alpha\in I}$ be a (possibly infinite) family of open subsets of $\mathbb{R}^d$. Then $\bigcup_{\alpha\in I}V_\alpha$ is open.
		\item Let $V_1,...,V_n\in\mathbb{R}^d$ be open. Then $V_1\cap...\cap V_n$ is open.
	\end{enumerate}
}

\begin{proof}
	TODO: prove the above statements
\end{proof}

\ex{}{The set $U = \{(x_1,...,x_d)\in\mathbb{R}^d|x_d > 0\}$ is an open subset of $\mathbb{R}^d$.\\
\begin{myproof}
	Let $x\in U$ and set $\epsilon = x_d$. We then want to show that $B(x,\epsilon)\subset U$. Let $y\in B(x,\epsilon)$. Then $|x_d - y_d|\leq ||x-y|| = \sqrt{|x_1-y_1|^2+...+|x_d-y_d^2|}$
\end{myproof}}

\begin{note}
	The collection of open subsets of $\mathbb{R}^d$ is called the Standard Topology on $\mathbb{R}^d$.
\end{note}

\dfn{Convergence of Sequences in $\mathbb{R}^d$}{Let $(x_n)_{n\geq 0}$ be a sequence in $\mathbb{R}^d$. We say $\displaystyle x_n\to_x$ or $\displaystyle\lim_{n\to\infty}x_n=x$ if 
$$\forall\epsilon>0,\exists n_0\in\mathbb{R}\text{ such that }\forall n>n_0, d(x_n,x)<\epsilon$$
Note that $d(x_n,x)<\epsilon$ is equivalent to $x_n\in B(x,\epsilon)$.}

\ex{}{$\displaystyle\lim_{n\to\infty}\frac{1}{n} = 0$\\ TODO: Proof}

\thm{}{Let $x\in\mathbb{R}$ and $(x_n)_{n\geq 0}$ a sequence in $\mathbb{R}^d$. Then
$$\lim_{n\to\infty}x_n = x \iff \forall V \text{ nbhd of }x,\exists n_0\in\mathbb{R}\text{ such that } \forall n>n_0, x_n\in V$$}

\begin{myproof}
	$(\implies)$: Assume $\lim_{n\to\infty}x_n = x$. Let $V$ be a neighborhood of $x$. Then, by the definition of neighborhood, $\exists\epsilon>0$ such that $B(x,\epsilon)\subset V$. Then, by the definition of convergence, $\exists n_0\in\mathbb{R}$ such that $\forall n > n_0, x_n\in B(x,\epsilon)$. Then, since $B(x,\epsilon)\subset V$, we have $\forall n > n_0, x_n\in V$. Then, since $V$ is arbitrary, $\forall V$ a neighborhood of $x, \exists n_0\in\mathbb{R}, \forall n> n_0, x_n\in V$. Therefore,
	$$\lim_{n\to\infty}x_n = x \implies \forall V \text{ nbhd of }x,\exists n_0\in\mathbb{R}\text{ such that } \forall n>n_0, x_n\in V$$

	$(\impliedby)$ Assume $\forall V \text{ nbhd of }x,\exists n_0\in\mathbb{R}$ such that $ \forall n>n_0, x_n\in V$. Let $\epsilon > 0$. We know that $B(x,\epsilon)$ is a neighborhood of $x$. Then, by our assumption, $\exists n_0\in\mathbb{R}$ such that $\forall n>n_0, x\in B(x,\epsilon)$. Therefore, by the definition of convergence, $\lim_{n\to\infty}x_n = x$. Therefore,
	$$\lim_{n\to\infty}x_n = x \impliedby \forall V \text{ nbhd of }x,\exists n_0\in\mathbb{R}\text{ such that } \forall n>n_0, x_n\in V$$
\end{myproof}

\dfn{Continuity of Functions from $\mathbb{R}^d$ to $\mathbb{R}^k$}{A map $f:\mathbb{R}^d\to\mathbb{R}^k$ is \textbf{continuous at $x\in\mathbb{R}^d$} if 
$$\forall\epsilon > 0,\exists\delta > 0 \text{ such that } \forall y\in B(x,\delta), d(f(x),f(y))<\epsilon$$
Note that $d(f(x),f(y))<\epsilon$ is equivalent to $f(y)\in B(f(y),\epsilon)$, so $f$ is continuous if 
$$\forall\epsilon > 0,\exists\delta > 0 \text{ such that } \forall y\in B(x,\delta), f(y)\in B(f(y),\epsilon)$$
Another, equivalent definition is that $f$ is continuous if 
$$\forall\epsilon > 0,\exists\delta > 0 \text{ such that } f(B(x,\delta))\subset B(f(x),\epsilon)$$
We say that $f$ is \textbf{continuous} if it is continuous at every point.}

\ex{}{
\begin{enumerate}
	\item Let $v\in\mathbb{R}^d$ and let $f:\mathbb{R}^d\to\mathbb{R}^d$ defined by the rule $x\mapsto x+v$ is continuous on $\mathbb{R}^d$.\\
	Let $x\in\mathbb{R}^d$ and let $\epsilon>0$. Set $\delta = \epsilon$. We then want to show that $f(B(x,\delta))\subset B(f(x),\epsilon)$. Let $y\in B(x,\delta)$. Then 
	$$d(f(y),f(x)) = ||f(y)-f(x)|| = ||y+v-(x+v)|| = ||y-x||<\delta = \epsilon$$
	Therefore, $f(y)\in B(f(x),\epsilon)$.
	\item TODO: fill in remaining examples
\end{enumerate}
}

\thm{}{Let $f:\mathbb{R}^d\to\mathbb{R}^k$. Then the following statements hold
\begin{enumerate}
	\item $f$ is continuous at $x\in\mathbb{R}^d \iff \forall V$ neighborhood of $f(x), f^{-1}(V)$ is a neighborhood of $x$.
	\item $f$ is continuous on $\mathbb{R}^d \iff \forall U\subset\mathbb{R}^d$ open, $f^{-1}(U)\subset\mathbb{R}^k$ is open.
	\item $f$ is continuous at $x\in\mathbb{R}^d\iff\forall (x_n)_{n\geq 0}$ in $\mathbb{R}^d$ converging to $x$, $f(x_n)\to f(x)$ as $n\to \infty$. 
\end{enumerate}
}

\begin{myproof}
	TODO: complete proof of above statements
\end{myproof}

\dfn{Open Subset}{Let $X\subset\mathbb{R}^d$. A subset $U\subset X$ is \textbf{open in $X$} if $\forall x\in U,\exists\epsilon>0$ such that $B(x,\epsilon)\cap X\subset U$.}

A Topology of $X$, then, is a collection of subsets of $X$ that are open in $X$. This definition will be expanded and made rigorous later in the course.

\ex{Torus}{TODO: insert torus example here (preferable with illustration)}

\dfn{Homeomorphism}{Let $X,Y\subset\mathbb{R}^d$ be subsets. Then $X,Y$ are \textbf{homeomorphic} if $\exists f:X\to Y$ a bijection such that
$$\forall U\subset X, U\text{ is open in } X\iff f(U)\text{ open in } Y$$
The function $f$ is called a \textbf{homeomorphism}.}

\section{General Topologies}

\dfn{Topology}{Let $X$ be a set. A \textbf{topology} on $X$ is a collection $\mcT\subset\mathcal{P}(X)$ of subsets of $X$ called \textbf{open subsets} such that
\begin{enumerate}
	\item $\emptyset, X\in\mathcal{T}$
	\item Stability under unions. For any $(U_{\alpha})_{\alpha\in I}$ of elements of $\mathcal{T}$, $\bigcup_{\alpha\in I}U_{\alpha}$ is also an element of $\mathcal{T}$.
	\item Stability under finite intersection. For any $U_1,...,U_n\in\mathcal{T}$, we have $U_1\cap...\cap U_n\in\mathcal{T}$.
\end{enumerate}
The pair $(X,d)$ is called a \textbf{topological space}.
}


\dfn{\label{norm}Normed Linear Space and Norm $\boldsymbol{\|\cdot\|}$}{Let $V$ be a vector space over $\bbR$ (or $\bbC$). A norm on $V$ is function $\|\cdot\|\ V\to \bbR_{\geq 0}$ satisfying \begin{enumerate}[label=\bfseries\tiny\protect\circled{\small\arabic*}]
		\item \label{n:1}$\|x\|=0 \iff x=0$ $\forall$ $x\in V$
		\item \label{n:2}	$\|\lambda x\|=|\lambda|\|x\|$ $\forall$ $\lambda\in\bbR$(or $\bbC$), $x\in V$
		\item \label{n:3} $\|x+y\| \leq \|x\|+\|y\|$ $\forall$ $x,y\in V$ (Triangle Inequality/Subadditivity)
	\end{enumerate}And $V$ is called a normed linear space.

	$\bullet $ Same definition works with $V$ a vector space over $\bbC$ (again $\|\cdot\|\to\bbR_{\geq 0}$) where \ref{n:2} becomes $\|\lambda x\|=|\lambda|\|x\|$ $\forall$ $\lambda\in\bbC$, $x\in V$, where for $\lambda=a+ib$, $|\lambda|=\sqrt{a^2+b^2}$ }


\ex{$\bs{p-}$Norm}{\label{pnorm}$V={\bbR}^m$, $p\in\bbR_{\geq 0}$. Define for $x=(x_1,x_2,\cdots,x_m)\in\bbR^m$ $$\|x\|_p=\Big(|x_1|^p+|x_2|^p+\cdots+|x_m|^p\Big)^{\frac1p}$$(In school $p=2$)}
\textbf{Special Case $\bs{p=1}$}: $\|x\|_1=|x_1|+|x_2|+\cdots+|x_m|$ is clearly a norm by usual triangle inequality. \par
\textbf{Special Case $\bs{p\to\infty\ (\bbR^m$ with $\|\cdot\|_{\infty})}$}: $\|x\|_{\infty}=\max\{|x_1|,|x_2|,\cdots,|x_m|\}$\\
For $m=1$ these $p-$norms are nothing but $|x|$.
Now exercise
\qs{}{\label{exs1}Prove that triangle inequality is true if $p\geq 1$ for $p-$norms. (What goes wrong for $p<1$ ?)}
\sol{\textbf{For Property \ref{n:3} for norm-2}	\subsubsection*{\textbf{When field is $\bbR:$}} We have to show\begin{align*}
		         & \sum_i(x_i+y_i)^2\leq \left(\sqrt{\sum_ix_i^2} +\sqrt{\sum_iy_i^2}\right)^2                                       \\
		\implies & \sum_i (x_i^2+2x_iy_i+y_i^2)\leq \sum_ix_i^2+2\sqrt{\left[\sum_ix_i^2\right]\left[\sum_iy_i^2\right]}+\sum_iy_i^2 \\
		\implies & \left[\sum_ix_iy_i\right]^2\leq \left[\sum_ix_i^2\right]\left[\sum_iy_i^2\right]
	\end{align*}So in other words prove $\langle x,y\rangle^2 \leq \langle x,x\rangle\langle y,y\rangle$ where
	$$\langle x,y\rangle =\sum\limits_i x_iy_i$$

	\begin{note}
		\begin{itemize}
			\item $\|x\|^2=\langle x,x\rangle$
			\item $\langle x,y\rangle=\langle y,x\rangle$
			\item $\langle \cdot,\cdot\rangle$ is $\bbR-$linear in each slot i.e. \begin{align*}
				      \langle rx+x',y\rangle=r\langle x,y\rangle+\langle x',y\rangle	\text{ and similarly for second slot}
			      \end{align*}Here in $\langle x,y\rangle$ $x$ is in first slot and $y$ is in second slot.
		\end{itemize}
	\end{note}Now the statement is just the Cauchy-Schwartz Inequality. For proof $$\langle x,y\rangle^2\leq \langle x,x\rangle\langle y,y\rangle $$ expand everything of $\langle x-\lambda y,x-\lambda y\rangle$ which is going to give a quadratic equation in variable $\lambda $ \begin{align*}
		\langle x-\lambda y,x-\lambda y\rangle & =\langle x,x-\lambda y\rangle-\lambda\langle y,x-\lambda y\rangle                                       \\
		                                       & =\langle x ,x\rangle -\lambda\langle x,y\rangle -\lambda\langle y,x\rangle +\lambda^2\langle y,y\rangle \\
		                                       & =\langle x,x\rangle -2\lambda\langle x,y\rangle+\lambda^2\langle y,y\rangle
	\end{align*}Now unless $x=\lambda y$ we have $\langle x-\lambda y,x-\lambda y\rangle>0$ Hence the quadratic equation has no root therefore the discriminant is greater than zero.

	\subsubsection*{\textbf{When field is $\bbC:$}}Modify the definition by $$\langle x,y\rangle=\sum_i\overline{x_i}y_i$$Then we still have $\langle x,x\rangle\geq 0$}
\section{Open and Closed Ball}
\dfn{Open and Closed Ball in Normed Linear Space}{An open Ball of radius $r$ with center $x$ in Normed Linear Space $V$ is the set$$\{y\in V\mid \|x-y\|<r\}=B_r(x)$$and Closed ball is the set $$\{y\in V\mid\|x-y\|\leq r\}=\overline{B_r(x)}$$}


Now take $B_r(0)$ w.r.t \textcolor{myr}{$\bs{\|\cdot\|_1}$}, \textcolor{myg}{$\bs{\|\cdot\|_2}$}, \textcolor{myb}{$\bs{\|\cdot\|_{\infty}}$}.
%\begin{figure}[h]
%	\centering
%	\includegraphics[width=4cm]{images/1.png}
%\end{figure}
Now imagine a sequence converging to origin. So if I
\begin{center}
	\begin{tikzpicture}
		\draw (-2.5, 0) -- (2.5, 0) ;
		\draw (0, -2.5) -- (0, 2.5) ;
		\draw[myg,thick] (0,0) circle (2cm);
		\draw[myb,thick] (-2,2) -- (2,2) -- (2,-2) -- (-2,-2) -- cycle;
		\draw[myr,thick] (2,0) -- (0,-2) -- (-2,0) -- (0,2) --cycle;
	\end{tikzpicture}
\end{center}

draw an ordinary circle around the origin then no matter how small the circle the points of the sequence are eventually land inside the circle. If instead of that circle can same be said for diamond w.r.t norm 2. Then i can take circle that is inside that diamond. Same is true for $\infty-$norm. Hence convergence with respect to all norm $1$ and norm 2 and even $\infty$ results for convergence.






Now there is no reason why we can not consider a norm on an infinite dimensional vector space. It will work. Perhaps i can define only for some sequences where the morm converges. \ex{}{Suppose for set of all bounded infinite sequences a vector space because every number in a vector is less than some number so if you add two vectors then add the bound and if you scale then scale the bound. Now the $\infty$ norm works on that.

	Now suppose you take all continuous real valued functions on closed interval $[0,1]$, such a function is bounded and this is a vector space and we can define $\infty-$norm even for that because for all $f$ in this space attains its maximum value so just take that maximum value. Its an extremely infinite dimensional space.


}

\begin{note}
	$\bbR^{\infty}$ is the space of all sequences.
\end{note}

\qs{}{\label{exs2}Modify the above proof for field $\bbC$}
\qs{}{\label{exs3}Show that the following are normed linear spaces.\begin{enumerate}[label=(\alph*)]
		\item $l^{\infty}=$ Set of all bounded infinite sequences $(x_1,x_2,\cdots)$ $x_i\in\bbR$ with norm $\|x\|=\sup |x_i|$
		\item $C[0,1]=$ Set of all continuous functions $[0,1]\to \bbR$ with norm $\|f\|=\sup\limits_{x\in[0,1]}|f(x)|$
	\end{enumerate}
}
\section{Limit of a Sequence}
\dfn{Limit of Sequence in Normed Linear Space}{A sequence  $\{s_n\}$ in a normed linear space $V$ converge to $s$ means $\forall$ real number $\eps>0$ $\exists$ natural number $N$ such that for $\forall\ n>N$ $\|s-s_n\|<\eps$}
\section{Continuity}
\dfn{Continuity in Normed Linear Space}{Let $S$ be a subset of $V$ and $f:\ S\to W$ where $V,W$ are normed linear space. $f$ is continuous at $v\in V$ means $\forall$ $\eps>0$, $\exists$ $\delta>0$, st whenever $\|x-v\|<\delta$ for $x\in S$ one has $\|f(x)-f(v)\|<\eps$}
Distance in a normed linear space for $x,y\in V$ is $$d(x,y)=\|x,y\|$$ Hence properties of this $d$ are\begin{enumerate}[label=\bfseries\tiny\protect\circled{\small\arabic*}]
	\item[\ref{n:1}] \label{m:1} $d(x,y)=0 \iff x=y$
	\item[\ref{n:2}] \label{m:2} $d(\lambda x,\lambda y)=|\lambda|d(x,y)$ for any scalar $\lambda$
	\item[\ref{n:3}] \label{m:3} $d(u,v)+d(u,v)\geq d(u,w)$
\end{enumerate}
